\documentclass{article}

%%%%%%%%%%%%%%% LIBRERIAS %%%%%%%%%%%%%%%%%%%%%
\usepackage{amsmath}
\usepackage{titlesec}
\usepackage{titletoc}
\usepackage{graphicx}
\usepackage[spanish,es-tabla]{babel} % 'es-tabla' cambia Cuadro→Tabla
\usepackage{hyperref}                % cargar después de babel
\usepackage{float}
\usepackage{circuitikz}
\usepackage[a4paper, margin=3cm]{geometry}
%%%%%%%%%%%%%%%%%%% VARIABLES %%%%%%%%%%%%%%%%%%%%
\newcommand{\Facultad}{Instituto Tecnológico \\de\\ Buenos Aires} %constantes
\newcommand{\TPn}{Trabajo Práctico N° 1}
\newcommand{\TPtema}{Corriente Continua}
\renewcommand{\thesection}{\arabic{section}}          % 2
\renewcommand{\thesubsection}{\quad \alph{subsection}}   % a
\renewcommand{\thesubsubsection}{\quad \thesubsection.~\roman{subsubsection}} % a. i
\graphicspath{{imagenes&matlab/}} %para que acceda a las fotos en la carpeta directamente


%%%%%%%%%%%%%%%%%% FORMATO TÍTULO Y NUMERACIÓN %%%%%%%%%%%%%%%%%%%
% Numeración de secciones
\renewcommand{\thesection}{\arabic{section}.}          
\renewcommand{\thesubsection}{\thesection\arabic{subsection}}       
\renewcommand{\thesubsubsection}{$\alph{subsubsection})$}

% Numerar hasta subsubsecciones
\setcounter{secnumdepth}{3}

% Formato de títulos
\titleformat{\section}{\Huge\bfseries}{\thesection}{1em}{}
\titleformat{\subsection}{\LARGE\bfseries}{\thesubsection}{0.5em}{}
\titleformat{\subsubsection}{\large\bfseries}{\thesubsubsection}{0.5em}{}



%%%%%%%%%%%%%%%%%%% ARCHIVO %%%%%%%%%%%%%%%%%%%%%%%%
\begin{document}

    %%%CARATULA%%%
    \begin{titlepage} %creo portada

        \begin{flushleft}
            \centering
            \includegraphics[width=0.3\textwidth]{Logo_ITBA.png}
        \end{flushleft}

        \centering
            
        {\scshape\LARGE \Facultad \par} %\par sirve para indicar un final de parrafo
        \vspace{1cm}                    %esto hace un espacio entre lineas de 1cm


        {\huge\bfseries \TPn \par}
        \vspace{1.5cm}
        {\Large Teoría de Circuitos I\\ 25.10 \par}
        \vfill                      %sirve para rellenar el espacio y quede simétrico. Si se añaden otros, se dividen el espacio de forma equitativa
        {\Large \bfseries Grupo N° 2 \par}
        \vspace{1cm}
        {\large Juan Bautista Correa Uranga \hfill Legajo: 65016 \par} %\hfill sirve para hacerlo simétrico
        {\large Juan Ignacio Caorsi \hfill Legajo: 65532  \par}
        {\large Rita Moschini \hfill Legajo: 67026 \par} 
        \vfill
        {\large \today\par}
        \vfil

    \end{titlepage}


    %%%RESUMEN%%%
    {\centering \LARGE \bfseries Resumen \par}

    \newpage

    %%%INDICE%%%
    \tableofcontents %esto sirve para crear el índice
    \newpage

    %%%Introduccion%%%
    \section{Introducción}

        El objetivo principal de este trabajo practico fue utilizar nuestros conocimientos de corriente alterna para estudiar la relación entre la potencia activa, real y aparente, así como el efecto que tiene la introducción de un núcleo en un inductor.

        \subsection{Instrumental}

        En esta experiencia se utilizaron los siguientes instrumentos:

        \begin{itemize}
            \renewcommand{\labelitemi}{$\bullet$}
            \item {\bfseries Variac}: Fuente de tensión AC regulable, 50 Hz.
            \item {\bfseries Vatímetro analógico}: El mismo sirve para poder medir la potencia activa (P) consumido por el circuito.
                    Para calcular los valores, el mismo debe medir la tensión y la corriente del circuito. Para ello se conecta su voltímetro en paralelo con el circuito y su amperímetro en serie con el circuito (observar Figura \ref{fig:vatimetro}). \par

                    \begin{figure}[h!] % el [h!] fuerza a ponerlo aquí
                        \centering
                        \begin{tikzpicture}
                            % Paths, nodes and wires:
                            \node[oscillator] at (21.49, -15.26){};
                            \draw (21, -14) to[qpprobe] (26, -14);
                            \draw (21, -14) -- (21, -14.77);
                            \draw (21, -15.75) -| (21, -16.5) -- (26, -16.5);
                            \draw (26, -14) to[cute inductor] (26, -16.5);
                            \draw (23.248, -14.28) -| (23.25, -14.75) -- (22.5, -14.75) -| (22.5, -14);
                            \draw (23.752, -14.28) |- (23.75, -16.5);
                        \end{tikzpicture}
                        \caption{Esquema de conexionado de vatímetro}
                        \label{fig:vatimetro}
                    \end{figure}
            \item {\bfseries Amperímetro analógico}
            \item {\bfseries Voltímetro analógico}
            \item {\bfseries Inductor}: Consta de un armazón plástico hueco, al cual se le enrolla un alambre sucesivas veces. El mismo tiene asociado una resistencia y un valor L inductivo. 
                Este tenía un hueco en el medio donde se podía introducir un núcleo de material sólido. 
        \end{itemize}
        
        \subsection{Marco teórico}

        En esta práctica se usaron los siguientes conocimientos teóricos:

        \subsubsection{Inductor}

        Como se mencionó anteriormente, un inductor consta de un alambre enrollado. El mismo es capaz de almacenar energía eléctrica en forma de energía magnética.
         La ecuación del mismo viene dada por:
        \begin{equation*}
            V=L \frac{dI}{dT}
        \end{equation*}
        Donde V es la tensión, L es la constante inductiva en [H] e I es la corriente que circula por el inductor.

        Usando cambio de variable a números complejos, se puede llegar a la siguiente ecuación.
        \begin{equation}
            V=j\cdot \omega\cdot L \cdot I
            \label{induct}
        \end{equation}
        con $j$ la constante compleja, $\omega$ la frecuencia en [$\frac{rads}{seg}$], $L$ la constante inductiva en [H] y $I$ corriente [A].\par

        \subsubsection{Potencia compleja}

        Al analizar la potencia asociada a circuitos AC, se debe tener en consideración los siguientes aspectos.\par
        La potencia total se llama { \bfseries potencia compleja} $\vec{S}= V \cdot I^{*}$ [VA], y es la potencia total del sistema. El módulo de la misma, se llama {\bfseries potencia aparente} $S=|I_{rms}| \cdot |V_{rms}|$ [VA]. No obstante, este número no representa la potencia total usada por el sistema.
        La potencia compleja se puede describir de la siguiente forma $\vec{S}=P + j \cdot Q$, donde P es la { \bfseries potencia activa}  [W] y Q es la {\bfseries potencia reactiva} [VAR]. \par

        La potencia activa (P) es la potencia que es efectivamente usada por el sistema. Esta representa la parte real de la potencia compleja.
        Más aun, la misma se puede descomponer en $P= S \cdot \cos (\varphi)$. De esto se puede deducir que la misma es máxima cuando el desfasaje $\varphi = \theta_{V} - \theta_I $ es nulo. \par

        Por otra parte, la potencia reactiva es la parte imaginaria de $\vec{S}$. Esta representa la potencia inductiva o capacitiva. Esta se puede descomponer en $ Q = S \cdot \sen (\varphi)= \pm \frac{V_\text{rms}^2}{X}$ donde es (+) si es inductivo o (-) si es capacitivo.
        De esta formula se puede deducir que la misma es mínima cuando el desfasaje entre la corriente y la tensión se acerca a cero. La potencia reactiva, a diferencia de la potencia activa, es una potencia la cual no puede ser aprovechada por el sistema, 
        ya que la misma esta asociada (como se pudo observar en la ecuación que la define) a la energía almacenada por los inductores y los capacitores, es decir que idealmente no es disipada o es disipada en proporciones mínimas. \par

        En consecuencia, este último tipo de potencia se suele reducir al mínimo posible, por motivos económicos.

        Por último, P y Q se pueden relacionar usando el triangulo de potencia (observar Figura \ref{fig:triangulo_de_potencia}) mediante la fórmula $S^2=Q^2+P^2$. 
        \begin{figure}[h!] % h! = aquí, lo más forzado posible
            \centering
            \includegraphics[width=0.5\textwidth]{Trojkat-mocy.png} % nombre del archivo
            \caption{Triangulo de potencia inductivo}
            \label{fig:triangulo_de_potencia} % etiqueta para referencias
        \end{figure}

    %%%Desarrollo%%%
    \indent
    \section{Desarrollo}

        \subsection{Procedimiento}

            Para esta experiencia se conectaron los instrumentos de medición junto con el variac y el inductor de la siguiente forma (Observar Figura: \ref{fig:circuito-inductor}). Luego se setió el variac a 120 V. Finalmente se procedió a la toma de mediciones de tensión, de corriente y la potencia usando el instrumental . 
            \begin{figure}[h!]
                \centering
                \begin{tikzpicture}
                    % Paths, nodes and wires:
                    \draw (22.5, -14.5) to[voltmeter] (22.5, -17.25);
                    \draw (24.25, -17.25) to[cute inductor] (24.25, -14.5);
                    \draw (17.75, -14.5) to[ammeter] (20.25, -14.5);
                    \draw (20.25, -14.5) to[qpprobe] (21.75, -14.5);
                    \draw (21.75, -14.5) -- (22.5, -14.5);
                    \draw (17.75, -14.5) to[sinusoidal voltage source, l={$V_s$}] (17.75, -17.25);
                    \draw (22.5, -14.5) -- (24.25, -14.5);
                    \draw (17.75, -17.25) -- (24.25, -17.25);
                \end{tikzpicture}
                \caption{Circuito con fuente senoidal, amperímetro, voltímetro e inductor}
                \label{fig:circuito-inductor}
            \end{figure}

            Este procedimiento se realizo tres veces: 1) Con el nucleo del inductor vacío. 2) Una barra parcialmente introducida en el nucleo. 3) Con la barra totalmente introducida en el nucleo.

        \subsection{Mediciones}
            
            A continuación se presentan las mediciones tomadas en cada una de las tres instancias del experimento.

            \subsubsection{Núcleo vacío}

            \begin{table}[H]
                \centering
                \begin{tabular}{|c|c|c|c|}
                    \hline
                    $V_{rms} $[V] & $I_{rms} $[A] & $P $[W]  \\ \hline
                    120           & 1.175         & 29     \\ \hline
                \end{tabular}
                \caption{Mediciones con núcleo vacío}
                \label{tab:mediciones-nucleo-vacio}
            \end{table}

            \subsubsection{Núcleo de hierro parcialmente introducido}

            \begin{table}[H]
                \centering
                \begin{tabular}{|c|c|c|}
                    \hline
                    $V_{rms} $[V] & $I_{rms} $[A] & $P $[W] \\ \hline
                    122           & 0.6         & 16    \\ \hline
                \end{tabular}
                \caption{Mediciones con núcleo parcialmente introducido de hierro}
                \label{tab:mediciones-nucleo-parcialmente-introducido-hierro}
            \end{table}

            \subsubsection{Núcleo de hierro totalmente introducido}

            \begin{table}[H]
                \centering
                \begin{tabular}{|c|c|c|c|}
                    \hline
                    $V_{rms} $[V] & $I_{rms} $[A] & $P $[W] \\ \hline
                    124           & 0.25         & 12    \\ \hline
                \end{tabular}
                \caption{Mediciones con núcleo totalmente introducido de hierro}
                \label{tab:mediciones-nucleo-totalmente-introducido-hierro}
            \end{table}

            \subsubsection{Núcleo de aluminio parcialmente introducido}

            \begin{table}[H]
                \centering
                \begin{tabular}{|c|c|c|c|}
                    \hline
                    $V_{rms} $[V] & $I_{rms} $[A] & $P $[W] \\ \hline
                    122           & 0.575         & 10    \\ \hline
                \end{tabular}
                \caption{Mediciones con núcleo parcialmente introducido de aluminio}
                \label{tab:mediciones-nucleo-parcialmente-introducido-aluminio}
            \end{table}

            \subsubsection{Núcleo de aluminio totalmente introducido}

            \begin{table}[H]
                \centering
                \begin{tabular}{|c|c|c|c|}
                    \hline
                    $V_{rms} $[V] & $I_{rms} $[A] & $P $[W] \\ \hline
                    124           & 0.25         & 6    \\ \hline
                \end{tabular}
                \caption{Mediciones con núcleo totalmente introducido de aluminio}
                \label{tab:mediciones-nucleo-totalmente-introducido-aluminio}
            \end{table}

        \subsection{Cálculos} \label{sec:Cálculos}

            \subsubsection*{Fórmulas utilizadas}

            A partir de las magnitudes medidas de potencia activa $P$ (en watts), 
            tensión $V$ (en volts) y corriente $I$ (en amperes), se calcularon las 
            siguientes variables eléctricas:

            \begin{enumerate}
                \item \textbf{Potencia aparente:}
                \[
                    S = V \cdot I
                \]
                donde $S$ se mide en volt–amperes (VA).

                \item \textbf{Potencia reactiva:}
                \[
                    Q = \sqrt{S^{2} - P^{2}}
                \]
                expresada en volt–amperes reactivos (VAR). \par
                
                La ecuación presentada aquí permite calcular el módulo de la potencia reactiva, pero como se dijo
                anteriormente esta puede tomar valores tanto positivos como negativos. En este trabajo práctico, 
                tomamos $Q=|Q|$ a sabiendas de que, dado que el único componente pasivo de nuestro circuito es inductivo, 
                la potencia reactiva sera inductiva también.

                \item \textbf{Factor de potencia:}
                \[
                    \cos \varphi = \frac{P}{S}
                \]

                \item \textbf{Ángulo de desfase:}
                \[
                    \varphi = \arccos\left( \frac{P}{S} \right)
                \]
                donde $\varphi$ se expresa en grados al convertir el resultado de 
                radianes:
                \[
                    \varphi \,[^\circ] = \arccos\left( \frac{P}{S} \right) \cdot 
                    \frac{180}{\pi}
                \]
            \end{enumerate}

                    Estas expresiones permiten representar el \textit{triángulo de potencias}, 
                    donde la potencia activa $P$ se ubica en el eje horizontal, la potencia 
                    reactiva $Q$ en el eje vertical, y la potencia aparente $S$ corresponde a 
                    la hipotenusa.\par

                Por otra parte, se calculó el valor de L, considerando la resistencia interna del inductor mediante la siguiente ecuación:

                \begin{equation*}
                    L = \frac{1}{\omega} \cdot \sqrt{\frac{V^2}{I^2}-R^2}
                \end{equation*}
                con $L$ la constante del inductor, $\omega$ la frecuencia, $V$ la tensión medida, $I$ la corriente medida y $R$ la resistencia interna del inductor. \par
                
                    A continuación se presentan los triángulos de potencia de cada una de las configuraciones estudiadas, y en adición,
                    la configuración con núcleo totalmente introducido de aluminio. 

                    \subsubsection{Núcleo vacío}
                        Se obtuvieron los siguientes resultados:
                        \begin{itemize}
                            \item Potencia aparente (S): 141 VA
                            \item Potencia reactiva (Q): 137,98 VAR
                            \item Factor de potencia (cos $\phi$): 0,206
                            \item Ángulo de desfase $\phi$: 78,131°
                            \item Constante del inductor (L): 316,208 mH
                        \end{itemize}

                        \begin{figure}[H]
                            \centering
                            \includegraphics[width=0.8\textwidth]{graficoAire.png}
                            \caption{Triángulo de Potencias para el circuito con núcleo de aire.}
                            \label{fig:graficoAire}
                        \end{figure}
                    
                    \subsubsection{Núcleo de hierro parcialmente introducido}
                        
                        Se obtuvieron los siguientes resultados:
                        \begin{itemize}
                            \item Potencia aparente (S): 73,2 VA
                            \item Potencia reactiva (Q): 71,43 VAR
                            \item Factor de potencia (cos $\phi$): 0.218
                            \item Ángulo de desfase $\phi$: 73,37°
                            \item Constante del inductor (L): 642,82 mH
                        \end{itemize}

                        \begin{figure}[H]
                            \centering
                            \includegraphics[width=0.8\textwidth]{graficoParcialHierro.png}
                            \caption{Triángulo de Potencias para el circuito con núcleo parcialmente introducido de hierro.}
                            \label{fig:graficoParcialHierro}
                        \end{figure}


                    \subsubsection{Núcleo de hierro totalmente introducido}

                        Se obtuvieron los siguientes resultados:
                        \begin{itemize}
                            \item Potencia aparente (S): 31.00 VA
                            \item Potencia reactiva (Q): 28.58 VAR
                            \item Factor de potencia (cos $\phi$): 0.387
                            \item Ángulo de desfase $\phi$: 67.23°
                            \item Constante del inductor (L): 1577 mH
                        \end{itemize}

                        \begin{figure}[H]
                            \centering
                            \includegraphics[width=0.8\textwidth]{graficoTotalHierro.png}
                            \caption{Triángulo de Potencias para el circuito con núcleo totalmente introducido de hierro.}
                            \label{fig:graficoTotalHierro}
                        \end{figure}


                        \subsubsection{Núcleo de aluminio parcialmente introducido}

                            Se obtuvieron los siguientes resultados:
                        \begin{itemize}
                            \item Potencia aparente (S): 70,15 VA
                            \item Potencia reactiva (Q): 63,43 VAR
                            \item Factor de potencia (cos $\phi$): 0,142
                            \item Ángulo de desfase $\phi$: 81,80°
                            \item Constante del inductor (L): 671,14 mH
                        \end{itemize}

                %     \begin{figure}[H]
                    %        \centering
                %           \includegraphics[width=0.8\textwidth]{graficoParcialAluminio.png}
                %           \caption{Triángulo de Potencias para el circuito con núcleo parcialmente introducido de aluminio.}
                %          \label{fig:graficoParcialAluminio}
                %       \end{figure}
                    

                    \subsubsection{Núcleo de aluminio totalmente introducido}

                            Se obtuvieron los siguientes resultados:
                        \begin{itemize}
                            \item Potencia aparente (S): 31.00 VA
                            \item Potencia reactiva (Q): 30.41 VAR
                            \item Factor de potencia (cos $\phi$): 0.194
                            \item Ángulo de desfase $\phi$: 78.84°
                            \item Constante del inductor (L): 1577 mH
                        \end{itemize}

                        \begin{figure}[H]
                            \centering
                            \includegraphics[width=0.8\textwidth]{graficoTotalAluminio.png}
                            \caption{Triángulo de Potencias para el circuito con núcleo totalmente introducido de aluminio.}
                            \label{fig:graficoTotalAluminio}
                        \end{figure}
            
                        
        \subsection{Análisis}
        
            Se puede ver en los resultados de la sección \ref{sec:Cálculos} o en la tabla \ref{tab:ConstantesDelInductor} que el valor de la inductancia L de la bobina varía acorde considerablemente en función no del material del núcleo, sino según qué tan introducido está el mismo dentro del inductor. Dado que la inductancia de la bobina aporta a la impedancia del circuito, este comportamiento se ve reflejado en todas las mediciones indirectas; esto se puede observar en la sección \ref{sec:Cálculos}, o en las tablas \ref{tab:PotenciasAparentes} y \ref{tab:PotenciasComplejas}.
            
             \begin{table}[H]
                \centering
                \begin{tabular}{|l|c|c|c|}
                \hline
                Constante del inductor               & Aire (sin núcleo)     & Núcleo de hierro & Núcleo de aluminio \\ \hline
                                                    & 316,208 mH             &                  &                     \\ \hline
                Núcleo parcialmente introducido     &                        & 642,82 mH        & 671,14 mH            \\ \hline
                Núcleo totalmente introducido       &                        & 1577 mH          & 1577 mH                \\ \hline
                \end{tabular}
                \caption{Constantes de inductores medidas con diferentes núcleos}
                \label{tab:ConstantesDelInductor}
            \end{table}

            
            \begin{table}[H]
                \centering
                \begin{tabular}{|l|c|c|c|}
                \hline
                Potencia aparente                    & Aire (sin núcleo)  & Núcleo de hierro & Núcleo de aluminio \\ \hline
                                                    & 141 VA             &                  &                     \\ \hline
                Núcleo parcialmente introducido     &                    &  73,2 VA         &  70,15 VA           \\ \hline
                Núcleo totalmente introducido       &                    &  31.00 VA        &  31.00 VA           \\ \hline
                \end{tabular}
                \caption{Potencias aparentes  en diferentes configuraciones y núcleos.}
                \label{tab:PotenciasAparentes}
            \end{table}


            \begin{table}[H]
                \centering
                \begin{tabular}{|l|c|c|c|}
                \hline
                Potencia compleja                    & Aire (sin núcleo)  & Núcleo de hierro & Núcleo de aluminio \\ \hline
                                                    &  137,98 VAR        &                  &                     \\ \hline
                Núcleo parcialmente introducido     &                    &  71,43 VAR       &  63,43 VAR          \\ \hline
                Núcleo totalmente introducido       &                    &   28.58 VAR      &  30.41 VAR           \\ \hline
                \end{tabular}
                \caption{Potencias complejas en diferentes configuraciones y núcleos.}
                \label{tab:PotenciasComplejas}
            \end{table}


    %%%Conclusiones%%%
    \section{Conclusiones}

    %%%Anexos%%%
    \section{Anexos}

\end{document}
