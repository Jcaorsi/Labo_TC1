
\documentclass{article}

%%%%%%%%%%%%%%% LIBRERIAS %%%%%%%%%%%%%%%%%%%%%
\usepackage{amsmath}
\usepackage{titlesec}
\usepackage{titletoc}
\usepackage{graphicx}
\usepackage[spanish,es-tabla]{babel} % 'es-tabla' cambia Cuadro→Tabla
\usepackage{hyperref}                % cargar después de babel
\usepackage{float}
\usepackage{circuitikz}
%%%%%%%%%%%%%%%%%%% VARIABLES %%%%%%%%%%%%%%%%%%%%
\newcommand{\Facultad}{Instituto Tecnológico \\de\\ Buenos Aires} %constantes
\newcommand{\TPn}{Trabajo Práctico N° 1}
\newcommand{\TPtema}{Corriente Continua}
\renewcommand{\thesection}{\arabic{section}}          % 2
\renewcommand{\thesubsection}{\quad \alph{subsection}}   % a
\renewcommand{\thesubsubsection}{\quad \thesubsection.~\roman{subsubsection}} % a. i
\graphicspath{{imagenes&matlab/}} %para que acceda a las fotos en la carpeta directamente


%%%%%%%%%%%%%%%%%% FORMATO TÍTULO Y NUMERACIÓN %%%%%%%%%%%%%%%%%%%
% Numeración de secciones
\renewcommand{\thesection}{\arabic{section}.}          
\renewcommand{\thesubsection}{\thesection\arabic{subsection}}       
\renewcommand{\thesubsubsection}{$\alph{subsubsection})$}

% Numerar hasta subsubsecciones
\setcounter{secnumdepth}{3}

% Formato de títulos
\titleformat{\section}{\Huge\bfseries}{\thesection}{1em}{}
\titleformat{\subsection}{\LARGE\bfseries}{\thesubsection}{0.5em}{}
\titleformat{\subsubsection}{\large\bfseries}{\thesubsubsection}{0.5em}{}



%%%%%%%%%%%%%%%%%%% ARCHIVO %%%%%%%%%%%%%%%%%%%%%%%%
\begin{document}

    %%%CARATULA%%%
    \begin{titlepage} %creo portada

        \begin{flushleft}
            \centering
            \includegraphics[width=0.3\textwidth]{Logo_ITBA.png}
        \end{flushleft}

        \centering
            
        {\scshape\LARGE \Facultad \par} %\par sirve para indicar un final de parrafo
        \vspace{1cm}                    %esto hace un espacio entre lineas de 1cm


        {\huge\bfseries \TPn \par}
        \vspace{1.5cm}
        {\Large Teoría de Circuitos I\\ 25.10 \par}
        \vfill                      %sirve para rellenar el espacio y quede simétrico. Si se añaden otros, se dividen el espacio de forma equitativa
        {\Large \bfseries Grupo N° 5 \par}
        \vspace{1cm}
        {\large Juan Bautista Correa Uranga \hfill Legajo: 65016 \par} %\hfill sirve para hacerlo simétrico
        {\large Juan Ignacio Caorsi \hfill Legajo: 65532  \par}
        {\large Rita Moschini \hfill Legajo: 67026 \par} 
        \vfill
        {\large \today\par}
        \vfil

    \end{titlepage}


    %%%RESUMEN%%%
    {\centering \LARGE \bfseries Resumen \par}

    \newpage

    %%%INDICE%%%
    \tableofcontents %esto sirve para crear el índice
    \newpage

    %%%Introduccion%%%
    \section{Introducción}

    

        \subsection{Subseccion 1}


        \begin{figure}[H]
            \centering
            \begin{circuitikz}
            % Resistencias en serie
            \draw (0,0) to[R=$R_1$] (2,0)
                    to[R=$R_2$] (4,0);

            % símbolo =>
            \node at (5,0) {$\Rightarrow$};

            % Resistencia equivalente
            \draw (6,0) to[R=$R_{eq}$] (8,0);
            \end{circuitikz}
            \caption{Equivalencia de dos resistencias en serie}
              \label{fig:Resis_equiv_1}
        \end{figure}

        \begin{figure}[H]
            \centering
            \begin{circuitikz}
            \draw (0,0) to[short] (0,2)
                    to[R=$R_1$] (2,2)
                    -- (2,0)
                    to[R=$R_2$] (0,0);
            \node at (3,1) {$\Rightarrow$};
            \draw (4,0) to[short] (4,2)
                    to[R=$R_{eq}$] (6,2)
                    -- (6,0) -- (4,0);
            \end{circuitikz}
            \caption{Equivalencia de dos resistencias en paralelo}
            \label{fig:Resis_equiv_2}
        \end{figure}

        \subsection{subseccion 2}

      
        \subsection{subseccion 3}



    %%%Desarrollo%%%
    \indent
    \section{Desarrollo}
   
    %%%Conclusiones%%%
    \section{Conclusiones}

    %%%Anexos%%%
    \section{Anexos}

\end{document}