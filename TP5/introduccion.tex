\subsection{Instrumental}
        En esta experiencia se utilizaron los siguientes instrumentos:
\begin{itemize}
  \item \href{https://www.rftesolutions.com/index.php?main_page=product_info&products_id=729}{Osciloscopio Keysight (Agilent) DSO6014A}
  \item \href{https://www.keysight.com/us/en/product/EDU33212A/waveform-generator-20mhz-2-channel.html}{Generador de ondas}  con 
        resistencia interna de $ 50 \Omega $
  \item Analizador de impedancias.
  \item Resistencia de $ 100 \Omega $ (nominal y medido).
  \item Capacitor de 2,2 nF.
  \item Inductor de inductancia entre 0,5 mH y 2 mH
\end{itemize}

\subsection{Marco teórico}
Un circuito RLC serie está compuesto por una resistencia de valor R, un inductor de inductancia L y un capacitor de capacitancia C 
conectados en serie. La frecuencia de resonancia para un circuito RLC, serie o paralelo, es
\begin{equation*}
    w_0 = \frac{1}{\sqrt{LC}}
\end{equation*}

Por otro lado, se puede calcular el factor de calidad o factor de selectividad Q
\begin{equation*}
    Q_S = \frac{\omega_R L}{R} \quad\quad
    Q_P = \frac{\omega_R}{RC}
\end{equation*}
Cuando mayor sea Q, más "angosta" será la curva de la tensión en función de la frecuencia, es decir se amortiguarán más frecuencias.

FALTA PONER COMO CALCULAMOS EL ACNHO DE BANDA, PERO QUIERO QUE ESTE DEFINIDO ANTES DE PONERLO

\par FALTA PONER LAS FUNCIONES TRANSFERENCIA DE LOS DIAGRAMAS DE BODE, PERO QUIERO HACER UN POCO DE LA GUIA PRIMERO

\par LO DE LA SEÑAL TRIANGULAR DIRECTAMENTE NO LO ENTENDI