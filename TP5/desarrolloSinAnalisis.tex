\graphicspath{{imagenes/}} %para que acceda a las fotos en la carpeta directamente

\subsection{Procedimiento}

    \subsubsection{Analizador de impedancias}
    Con el fin de saber entre cuáles frecuencias se encontraría la frecuencia de resonancia, dados el rango de inductancia de la bobina 
    utilizada, se obtuvieron sus valores teóricos para los extremos del rango de inductancia (0,5 mH y 2 mH), haciendo

    \begin{equation*}
        f_{1,-1} = \frac{1}{\sqrt{LC}} \cdot \frac{1}{2\pi}
    \end{equation*}

    También se obtuvo $ f_{-2} = \frac{f_{-1}}{2} $ (una octava menor a $f_{-1}$) y $  f_2 = f_1 \cdot 2 $ (una octava mayor a $ f_1 $)
    y se midieron la inductancia y la resistencia de la bobina para cada una de estas frecuencias, variando la frecuencia con el analizador
    de impedancias.

    \subsubsection{Resonancia}
        Se armó el circuito de la figura \ref{fig: circuito}, fijando la tensión del generador en 1 V de amplitud ( es decir, 2 V pico a
        pico), y conectando los canales del osciloscopio en la entrada y la salida de la resistencia, de manera que se pudiera observar
        la caída de tensión en la misma usando la función Math.

        \begin{figure}[H]
        \centering
            \begin{tikzpicture}
                \node[oscillator, xscale=0.8, yscale=0.8] at (3.892, 3.892){};
                \draw (4, 5) to[american resistor, /tikz/circuitikz/bipoles/length=1.12cm] (6, 5);
                \draw (6.5, 5) to[cute inductor] (8.5, 5);
                \draw (8.75, 4.75) to[capacitor, /tikz/circuitikz/bipoles/length=1.12cm] (8.75, 2.75);
                \draw (3.5, 4.284) -- (3.5, 5) -- (4, 5);
                \draw (6, 5) -- (7, 5);
                \draw (8.5, 5) -- (8.75, 5) -- (8.75, 4.25);
                \draw (8.75, 2.75) -- (3.5, 2.75) -- (3.5, 3.5);
                \node[shape=rectangle, minimum width=0.715cm, minimum height=0.465cm] at (2.75, 4){} node[anchor=north west, align=left, text width=0.327cm, inner sep=6pt] at (2.375, 4.25){Vs};
                \node[shape=rectangle, minimum width=0.715cm, minimum height=0.465cm] at (5, 5.5){} node[anchor=north west, align=left, text width=0.327cm, inner sep=6pt] at (4.625, 5.75){R};
                \node[shape=rectangle, minimum width=0.715cm, minimum height=0.465cm] at (7.55, 5.591){} node[anchor=north west, align=left, text width=0.327cm, inner sep=6pt] at (7.175, 5.841){L};
                \node[shape=rectangle, minimum width=0.715cm, minimum height=0.465cm] at (9.368, 3.854){} node[anchor=north west, align=left, text width=0.327cm, inner sep=6pt] at (8.993, 4.104){C};
                \draw (3.5, 5) -- (3.25, 5.25);
                \draw (6, 5) -| (6, 5.25);
                \node[shape=rectangle, minimum width=0.715cm, minimum height=0.465cm] at (2.987, 5.54){} node[anchor=north west, align=left, text width=0.327cm, inner sep=6pt] at (2.612, 5.79){Ch1};
                \node[shape=rectangle, minimum width=0.715cm, minimum height=0.465cm] at (5.875, 5.604){} node[anchor=north west, align=left, text width=0.327cm, inner sep=6pt] at (5.5, 5.854){Ch2};
            \end{tikzpicture}
            \caption{Circuito estudiado a lo largo de la práctica.}
            \label{fig: circuito}
        \end{figure}

        De esta manera, se varió la frecuencia de la señal generada entre $ f_{-1} $ y $ f_1 $, considerando que la frecuencia de
        resonancia se correspondía con la mayor caída de tensión sobre la resistencia, es decir cuando la señal ilustrada por Math era
        máxima, la frecuencia correspondiente era la de resonancia.

        Luego se buscaron las frecuencias de corte $ f_{c1} $ y $ f_{c2} $, es decir las frecuencias para las cuales $ P = \frac{P_{max}}{2} $ o bien 
        $ V = \frac{V_{max}}{\sqrt{2}} $ (siendo V la tensión pico en la resistencia) y se calculó el ancho de banda
        \begin{equation*}
            B = \frac{ f_{c1} + f_{c2}}{2}
        \end{equation*}

    \subsubsection{Respuesta en frecuencia}

\subsection{Datos recolectados}
    \subsubsection{Analizador de impedancias}
        \begin{table}[h!]
            \centering
            \begin{tabular}{|c|c|c|c|c|c|}
            \hline
                             & 38 kHz $(f_{-2})$  & 76 kHz $(f_{-1})$  & 114 kHz $(f_m)$  & 151 kHz $(f_1)$  & 304 kHz $(f_2)$      \\ \hline
            $ R_L [\Omega]$  & 60,9               & 99,2               &  126             & 150              & 257                 \\ \hline
            L  [mH]          & 1,562              & 1,414              & 1,32             & 1,257            & 1,162                \\ \hline
            \end{tabular}
            \caption{Mediciones de las impedancias de la bobina en función de la frecuencia, tomadas con el analizador de impedancias.}
        \end{table}


    \subsubsection{Resonancia}
        \begin{itemize}
            \item $ f_R= 80 kHz $
            \item $ f_{c1} = 63 kHz $
            \item $ f_{c2} = 102,5 kHz $
            \item $ V_{max} = 350,35 mV $
        \end{itemize}

    \subsubsection{Respuesta en frecuencia}
        \begin{figure}[H]
            \centering
            \includegraphics[width=0.5\linewidth]{Rbode.png}
            \caption{Diagrama de bode de la resistencia, obtenido con el osciloscopio.}
            \label{fig: Rbode}
        \end{figure}

        \begin{figure}[H]
            \centering
            \includegraphics[width=0.5\linewidth]{Lbode.png}
            \caption{Diagrama de bode del inductor, obtenido con el osciloscopio.}
            \label{fig: Lbode}
        \end{figure}

        \begin{figure}[H]
            \centering
            \includegraphics[width=0.5\linewidth]{Cbode.png}
            \caption{Diagrama de bode del capacitor, obtenido con el osciloscopio.}
            \label{fig: Cbode}
        \end{figure}

\subsection{Cálculos}