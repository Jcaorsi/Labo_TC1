\documentclass{article}

%%%%%%%%%%%%%%% LIBRERIAS %%%%%%%%%%%%%%%%%%%%%
\usepackage{amsmath}
\usepackage{titlesec}
\usepackage{titletoc}
\usepackage{graphicx}
\usepackage[spanish,es-tabla]{babel} % 'es-tabla' cambia Cuadro→Tabla
\usepackage{hyperref}                % cargar después de babel
\usepackage{float}
\usepackage{circuitikz}
%%%%%%%%%%%%%%%%%%% VARIABLES %%%%%%%%%%%%%%%%%%%%
\newcommand{\Facultad}{Instituto Tecnológico \\de\\ Buenos Aires} %constantes
\newcommand{\TPn}{Trabajo Práctico N° 3}
\newcommand{\TPtema}{Respuesta Transitoria}
\renewcommand{\thesection}{\arabic{section}}          % 2
\renewcommand{\thesubsection}{\quad \alph{subsection}}   % a
\renewcommand{\thesubsubsection}{\quad \thesubsection.~\roman{subsubsection}} % a. i
\graphicspath{{imagenes/}} %para que acceda a las fotos en la carpeta directamente


%%%%%%%%%%%%%%%%%% FORMATO TÍTULO Y NUMERACIÓN %%%%%%%%%%%%%%%%%%%
% Numeración de secciones
\renewcommand{\thesection}{\arabic{section}.}          
\renewcommand{\thesubsection}{\thesection\arabic{subsection}}       
\renewcommand{\thesubsubsection}{$\alph{subsubsection})$}

% Numerar hasta subsubsecciones
\setcounter{secnumdepth}{3}

% Formato de títulos
\titleformat{\section}{\Huge\bfseries}{\thesection}{1em}{}
\titleformat{\subsection}{\LARGE\bfseries}{\thesubsection}{0.5em}{}
\titleformat{\subsubsection}{\large\bfseries}{\thesubsubsection}{0.5em}{}



%%%%%%%%%%%%%%%%%%% ARCHIVO %%%%%%%%%%%%%%%%%%%%%%%%
\begin{document}

%%%CARATULA%%%
\begin{titlepage} %creo portada

        \begin{flushleft}
            \centering
            \includegraphics[width=0.3\textwidth]{Logo_ITBA.png}
        \end{flushleft}

        \centering
            
        {\scshape\LARGE \Facultad \par} %\par sirve para indicar un final de parrafo
        \vspace{1cm}                    %esto hace un espacio entre lineas de 1cm


        {\huge\bfseries \TPn \par}
        \vspace{1.5cm}
        {\Large Teoría de Circuitos I\\ 25.10 \par}
        \vfill                      %sirve para rellenar el espacio y quede simétrico. Si se añaden otros, se dividen el espacio de forma equitativa
        {\Large \bfseries Grupo N° 2 \par}
        \vspace{1cm}
        {\large Juan Bautista Correa Uranga \hfill Legajo: 65016 \par} %\hfill sirve para hacerlo simétrico
        {\large Juan Ignacio Caorsi \hfill Legajo: 65532  \par}
        {\large Rita Moschini \hfill Legajo: 67026 \par} 
        \vfill
        {\large \today\par}
        \vfil

    \end{titlepage}

 %%%RESUMEN%%%
{\centering \LARGE \bfseries Resumen \par}

\newpage

%%%INDICE%%%
\tableofcontents %esto sirve para crear el índice

\newpage

%%%Introduccion%%%
\section{Introducción}
    Este trabajo práctico aborda la respuesta transitoria
     en circuitos RLC serie. Se busca aplicar conceptos 
     teóricos mediante mediciones reales, utilizando generador 
     de señales y osciloscopio. Se analiza cómo varían las
      respuestas al modificar resistencia, inductancia y 
      capacitancia, observando casos de subamortiguamiento,
       sobreamortiguamiento y amortiguamiento crítico. 
       El objetivo es comprender la dinámica temporal 
       de circuitos RLC y validar modelos teóricos
       con datos experimentales.
    \subsection{Instrumental}
        En esta experiencia se utilizaron los siguientes instrumentos:
\begin{itemize}
  \item Osciloscopio Keysight DSOX 1202G con generador de ondas integrado.
  \item Capacitores de $47\,\text{pF}$ y $470\,\text{pF}$.
  \item Resistencias de 220 $\Omega$ nominal y potenciómetro de 5 k$\Omega$ nominal.
  \item Inductor de resistencia indefinida, con inductancia aproximada de $1\,\text{mH}$.
  \item Multímetro UNI-T,  Standar Digital multimeter, modelo: UT39C.
\end{itemize}

    \subsection{Marco teórico}

        Un circuito RLC serie está compuesto por una resistencia (R),
         una 
        inductancia (L) y una capacitancia (C) conectadas en serie.
        Dependiendo de la resistencia del circuito, la respuesta transitoria
         puede ser subamortiguada, sobreamortiguada o críticamente 
         amortiguada. La ecuación diferencial que describe la 
         respuesta del circuito RLC serie es:
        \begin{equation}
            L\frac{d^2i(t)}{dt^2} + R\frac{di(t)}{dt} + \frac{1}{C}i(t) = V_{in}(t)
        \end{equation}

        y su solución general es la suma de la respuesta natural y la respuesta forzada:

Para la respuesta natural (\( V_{\text{in}}(t) = 0 \)), se define:

\begin{equation}
\alpha = \frac{R}{2L}, \quad \omega_0 = \frac{1}{\sqrt{LC}}, \quad \Delta = \alpha^2 - \omega_0^2
\end{equation}

La solución general depende del valor de \( \Delta \):

\begin{itemize}
  \item \textbf{Sobreamortiguado} (\( \Delta > 0 \)):

  \begin{equation}
  i(t) = A e^{(-\alpha + \sqrt{\Delta})t} + B e^{(-\alpha - \sqrt{\Delta})t}
  \end{equation}

  \item \textbf{Críticamente amortiguado} (\( \Delta = 0 \)):

  \begin{equation}
  i(t) = (A + Bt)e^{-\alpha t}
  \end{equation}

  \item \textbf{Subamortiguado} (\( \Delta < 0 \)):

  \begin{equation}
  i(t) = e^{-\alpha t} \left( A \cos(\omega_d t) + B \sin(\omega_d t) \right)
  \end{equation}

  donde \( \omega_d = \sqrt{\omega_0^2 - \alpha^2} \).
\end{itemize}

%%%Desarrollo%%%
\section{Desarrollo}
    \subsection{Procedimiento}
    \subsection{Mediciones}
 
        \begin{itemize}
            \item $ R_f = 215 \Omega $
            \item $ R_{V_{max}} = 9980 \Omega $
            \item $ R_{V_{min}} = 2 \Omega $
            \item $ R_L = 0,8 \Omega $
            \item $ L \approx 1 mH $
        \end{itemize}
        
        \par
        \subsubsection*{Capacitor de C = 470 pF}
            \begin{itemize}
                \item Resistencia variable tal que el amortiguamiento fue crítico: $ R_{critico} = 1,9 k\Omega $ % item 1
                \item Tiempo $\tau$ en que la salida llegó a 3,175 V con $ R_V $ en su valor máximo: $ \tau = 5,75 \mu s$ % item 2
                \item Salida cuando $t=5\tau$ con $ R_V $ en su valor máximo: V=3,175 V % item 2
            \end{itemize}

        \par
        \subsubsection*{Capacitor de C = 47 pF}
            \begin{itemize}
                \item Resistencia variable tal que el amortiguamiento fue crítico: $ R_{critico} = 3,47 k\Omega $ %item 3
                \item Tiempo en que la salida llegó a 3,175 V con $ R_V $ en su valor crítico: $ t = 2,20 \mu s$ % item 3
                \item Tiempo en que la salida llegó a 5,24 V con $ R_V $ en su valor mínimo: $ t = 14,30 \mu s$ % item 4
                \item Tiempo en que la salida llegó a 4,982 V ($ 5V \pm 0,05V $) con las resistencias cortocircuitadas: $ t = 11 \mu s$ % item 5
            \end{itemize}

    \subsection{Cálculos}

        \subsection{Ecuaciones utilizadas}            

            \textbf{Cálculo del la resistencia variable tal que el amortiguamiento fuera crítico} \par \par
            $ \alpha_{serie} = \omega_0 \Rightarrow \frac{R}{2L} = \frac{1}{\sqrt{LC}}  $
            \begin{equation}
                R = \frac{2L}{\sqrt{LC}}
            \end{equation}  \par \par 

            \textbf{Cálculo de la inductancia} \par \par
            Tomando la resistencia variable tal que el amortiguamiento resultase crítico,
            $ R = \frac{2L}{\sqrt{LC}} = 2\frac{\sqrt{L}}{\sqrt{C}} $
            \begin{equation}
                L = \frac{C \cdot R^2}{4}
            \end{equation}

            
        \subsection{Resultados}

            \subsubsection*{Capacitor de C = 470 pF}
                0) Inductancia L
                \begin{equation}
                    L=0.505mH
                \end{equation}
                1) Resistencia variable tal que el amortiguamiento fuera crítico:
                \begin{equation}
                    R_{critico} = 1.857 k\Omega
                \end{equation} \par
                2) $ \tau $ con $ R_V = R_{V_{max}} = 9980 \Omega $: REVISAR
                 \begin{equation}
                    \tau = ??? \mu s 
                \end{equation}

            \subsubsection*{Capacitor de C = 47 pF}
                3) Resistencia variable tal que el amortiguamiento fuera crítico:
                \begin{equation}
                    R_{critico} = 6,34 k\Omega 
                \end{equation} \par
                4) Valor de $ \tau $ para $ R_V = R_{V_{min}} $: REVISAR
                \begin{equation}
                    \tau =  \mu s 
                \end{equation} \par
                6) Valor de $ \tau $ para $ R = R_L $ (resistencias cortocircuitadas): REVISAR
                \begin{equation}
                    \tau =  \mu s 
                \end{equation}
        


    \subsection{Análisis}
        \subsection*{Simulación}
        1) 


\section{Conclusiones}

\end{document}