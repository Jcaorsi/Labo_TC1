\graphicspath{{imagenes/}} %para que acceda a las fotos en la carpeta directamente

\subsection{Procedimiento}

	El procedimiento de este TP se dividió en 3 secciones:

	\begin{itemize}	
	\item En la primera sección se midieron los parámetros del inductor. Para ello, primero se midieron los
	 valores de inductancia de ambos lados del inductor. Eso realizo mediante el medidor LRC configurándolo 
	 a 50kHz. Seguido a eso, se conectó un generador de señal senoidal de 1,5 V pico a una frecuencia de
	 50 kHz a un bobinado y se midió la caída de tensión en ambos bobinados, para luego volver a realizar 
	 estas mediciones con el generador conectado al otro bobinado (observar figuras \ref{fig:circ1} y 
	 \ref{fig:circ2}).  	
	
		\begin{figure}[H]
    \centering
    \begin{minipage}[b]{0.45\textwidth}
        \centering
	\begin{tikzpicture}
	\node[transformer core, cute] at (16.55, -14.05){};
	\node[circ] at (16, -13.25){};
	\node[circ] at (17.25, -13.25){};
	\node[oscillator, xscale=0.71, yscale=0.71] at (12.49, -14.01){};
	\draw (12.142, -14.358) |- (15.5, -15.1);
	\draw (12.142, -13.662) |- (12.75, -13);
	\draw (17.6, -13) -| (19, -13);
	\draw (17.6, -15.1) -| (19, -15);
	\node[shape=rectangle, minimum width=0.965cm, minimum height=1.465cm] at (17.6, -15.1){} node[anchor=north west, align=left, text width=0.577cm, inner sep=6pt] at (17.1, -14.35){L2};
	\node[shape=rectangle, minimum width=1.215cm, minimum height=0.715cm] at (15.875, -14.75){} node[anchor=north west, align=left, text width=0.827cm, inner sep=6pt] at (15.25, -14.375){L1};
	\node[shape=rectangle, minimum width=0.965cm, minimum height=0.59cm] at (16, -12.25){} node[anchor=north west, align=left, text width=0.577cm, inner sep=6pt] at (15.5, -11.937){Ch1};
	\draw[-latex] (16, -12.563) -- (16, -13);
	\node[shape=rectangle, minimum width=1.715cm, minimum height=0.84cm] at (20.5, -13.188){} node[anchor=north west, align=left, text width=1.327cm, inner sep=6pt] at (19.625, -12.75){Ch2};
	\draw[-latex] (19.75, -13) -- (19, -13);
	\node[shape=rectangle, minimum width=0.465cm, minimum height=0.465cm](N1) at (12.74, -14.01){} node[anchor=center] at (N1.text){$V_s$} node[anchor=north west, align=left, text width=0.077cm, inner sep=6pt] at (12.49, -13.76){};
	\draw (12.75, -13) -- (15.5, -13);
	\end{tikzpicture}
        \caption{Generador conectado a L1}
        \label{fig:circ1}
    \end{minipage}
    \hfill

    \begin{minipage}[b]{0.45\textwidth}
        \centering
      \begin{tikzpicture}
		\node[transformer core, cute] at (16.55, -14.05){};
		\node[circ] at (16, -13.25){};
		\node[circ] at (17.25, -13.25){};
		\node[oscillator, xscale=0.71, yscale=0.71] at (12.49, -14.01){};
		\draw (12.142, -14.358) |- (15.5, -15.1);
		\draw (12.142, -13.662) |- (12.75, -13);
		\draw (17.6, -13) -| (19, -13);
		\draw (17.6, -15.1) -| (19, -15);
		\node[shape=rectangle, minimum width=0.965cm, minimum height=1.465cm] at (17.6, -15.1){} node[anchor=north west, align=left, text width=0.577cm, inner sep=6pt] at (17.1, -14.35){L1};
		\node[shape=rectangle, minimum width=1.215cm, minimum height=0.715cm] at (15.875, -14.75){} node[anchor=north west, align=left, text width=0.827cm, inner sep=6pt] at (15.25, -14.375){L2};
		\node[shape=rectangle, minimum width=0.965cm, minimum height=0.59cm] at (16, -12.25){} node[anchor=north west, align=left, text width=0.577cm, inner sep=6pt] at (15.5, -11.937){Ch1};
		\draw[-latex] (16, -12.563) -- (16, -13);
		\node[shape=rectangle, minimum width=1.715cm, minimum height=0.84cm] at (20.5, -13.188){} node[anchor=north west, align=left, text width=1.327cm, inner sep=6pt] at (19.625, -12.75){Ch2};
		\draw[-latex] (19.75, -13) -- (19, -13);
		\node[shape=rectangle, minimum width=0.465cm, minimum height=0.465cm](N1) at (12.74, -14.01){} node[anchor=center] at (N1.text){$V_s$} node[anchor=north west, align=left, text width=0.077cm, inner sep=6pt] at (12.49, -13.76){};
		\draw (12.75, -13) -- (15.5, -13);
	\end{tikzpicture}
        \caption{Generador conectado a L2}
        \label{fig:circ2}
    \end{minipage}
\end{figure}
	
	\item En la segunda parte se conectó una resistencia adicional $R_m$ = 47 $\Omega$ entre el generador y la entrada del transformador del lado elevador conocido en la primera parte de la práctica. Seguidamente, se varió el offset (se tomaron valores de -5v a 5v) de la señal usada en la primera sección y se midió la tensión de salida. La salida se considero como circuito abierto.

	
	\begin{figure}[H]
    \centering
	\begin{tikzpicture}
	\node[transformer core, cute] at (16.55, -14.05){};
	\node[circ] at (16, -13.25){};
	\node[circ] at (17.25, -13.25){};
	\node[oscillator, xscale=0.71, yscale=0.71] at (12.49, -14.01){};
	\draw (12.142, -14.358) |- (15.5, -15.1);
	\draw (15.5, -13) to[american resistor, /tikz/circuitikz/bipoles/length=1.11cm, l={$R_m$}] (12.75, -13);
	\draw (12.142, -13.662) |- (12.75, -13);
	\draw (17.6, -13) -| (19, -13);
	\draw (17.6, -15.1) -| (19, -15);
	\node[shape=rectangle, minimum width=0.965cm, minimum height=1.465cm] at (17.6, -15.1){} node[anchor=north west, align=left, text width=0.577cm, inner sep=6pt] at (17.1, -14.35){L2};
	\node[shape=rectangle, minimum width=1.215cm, minimum height=0.715cm] at (15.875, -14.75){} node[anchor=north west, align=left, text width=0.827cm, inner sep=6pt] at (15.25, -14.375){L1};
	\node[shape=rectangle, minimum width=0.965cm, minimum height=0.59cm] at (16, -12.25){} node[anchor=north west, align=left, text width=0.577cm, inner sep=6pt] at (15.5, -11.937){Ch1};
	\draw[-latex] (16, -12.563) -- (16, -13);
	\node[shape=rectangle, minimum width=1.715cm, minimum height=0.84cm] at (20.5, -13.188){} node[anchor=north west, align=left, text width=1.327cm, inner sep=6pt] at (19.625, -12.75){Ch2};
	\draw[-latex] (19.75, -13) -- (19, -13);
	\node[shape=rectangle, minimum width=0.465cm, minimum height=0.465cm](N1) at (12.74, -14.01){} node[anchor=center] at (N1.text){$V_s$} node[anchor=north west, align=left, text width=0.077cm, inner sep=6pt] at (12.49, -13.76){};
	\end{tikzpicture}
    \caption{Conexionado segunda parte}
    \label{fig:Circuito_segunda_parte}
	\end{figure}

	\item En la tercera parte se estudió la impedancia reflejada por el transformador. Para ello, manteniendo la resistencia adicional ($R_m$) de 47 $\Omega$ en el entrada del transformado y manteniendo la señal original sin offset, se añadió una resistencia de carga ($R_L$) a la salida del transformador de 1 $k\Omega$ (Observar figura \ref{fig:circp3.1}). Con esto, se midió la caída de tensión sobre la resistencia de 47 $\Omega$ y se calculó la corriente del generador. Con el cociente de la tensión de la fuente y la corriente, se obtuvo la impedancia vista por la fuente. Se repitió el mismo procedimiento, pero conectado al segundo bobinado (Observar figura \ref{fig:circp3.2}).
	\end{itemize}

\begin{figure}[H]
    \centering
	\begin{minipage}[b]{0.45\textwidth}
        \centering
        \begin{tikzpicture}
			\node[transformer core, cute] at (16.55, -14.05){};
			\node[circ] at (16, -13.25){};
			\node[circ] at (17.25, -13.25){};
			\node[oscillator, xscale=0.71, yscale=0.71] at (12.49, -14.01){};
			\draw (12.142, -14.358) |- (15.5, -15.1);
			\draw (15.5, -13) to[american resistor, /tikz/circuitikz/bipoles/length=1.11cm, l={$R_m$}] (12.75, -13);
			\draw (12.142, -13.662) |- (12.75, -13);
			\draw (17.6, -13) -| (19, -13);
			\draw (19, -13) to[american resistor, /tikz/circuitikz/bipoles/length=1.11cm, l={$R_L$}] (19, -15);
			\draw (17.6, -15.1) -| (19, -15);
			\node[shape=rectangle, minimum width=0.965cm, minimum height=1.465cm] at (16, -15){} node[anchor=north west, align=left, text width=0.577cm, inner sep=6pt] at (15.5, -14.25){L1};
			\node[shape=rectangle, minimum width=1.215cm, minimum height=0.715cm] at (17.5, -14.625){} node[anchor=north west, align=left, text width=0.827cm, inner sep=6pt] at (16.875, -14.25){L2};
			\node[shape=rectangle, minimum width=0.965cm, minimum height=0.59cm] at (13.25, -12.25){} node[anchor=north west, align=left, text width=0.577cm, inner sep=6pt] at (12.75, -11.937){Ch1};
			\draw[-latex] (13.25, -12.563) -- (13.25, -13);
			\node[shape=rectangle, minimum width=1.715cm, minimum height=0.84cm] at (15.375, -12.313){} node[anchor=north west, align=left, text width=1.327cm, inner sep=6pt] at (14.5, -11.875){Ch2};
			\draw[-latex] (15, -12.5) -- (15, -13);
			\node[shape=rectangle, minimum width=0.465cm, minimum height=0.465cm](N1) at (12.74, -14.01){} node[anchor=center] at (N1.text){$V_s$} node[anchor=north west, align=left, text width=0.077cm, inner sep=6pt] at (12.49, -13.76){};
		\end{tikzpicture}
        \caption{Generador conectado a L1}
        \label{fig:circp3.1}
    \end{minipage}
    \hfill

    \begin{minipage}[b]{0.45\textwidth}
        \centering
        \begin{tikzpicture}
		\node[transformer core, cute] at (16.55, -14.05){};
		\node[circ] at (16, -13.25){};
		\node[circ] at (17.25, -13.25){};
		\node[oscillator, xscale=0.71, yscale=0.71] at (12.49, -14.01){};
		\draw (12.142, -14.358) |- (15.5, -15.1);
		\draw (15.5, -13) to[american resistor, /tikz/circuitikz/bipoles/length=1.11cm, l={$R_m$}] (12.75, -13);
		\draw (12.142, -13.662) |- (12.75, -13);
		\draw (17.6, -13) -| (19, -13);
		\draw (19, -13) to[american resistor, /tikz/circuitikz/bipoles/length=1.11cm, l={$R_L$}] (19, -15);
		\draw (17.6, -15.1) -| (19, -15);
		\node[shape=rectangle, minimum width=0.965cm, minimum height=1.465cm] at (16, -15){} node[anchor=north west, align=left, text width=0.577cm, inner sep=6pt] at (15.5, -14.25){L2};
		\node[shape=rectangle, minimum width=1.215cm, minimum height=0.715cm] at (17.5, -14.625){} node[anchor=north west, align=left, text width=0.827cm, inner sep=6pt] at (16.875, -14.25){L1};
		\node[shape=rectangle, minimum width=0.965cm, minimum height=0.59cm] at (13.25, -12.25){} node[anchor=north west, align=left, text width=0.577cm, inner sep=6pt] at (12.75, -11.937){Ch1};
		\draw[-latex] (13.25, -12.563) -- (13.25, -13);
		\node[shape=rectangle, minimum width=1.715cm, minimum height=0.84cm] at (15.375, -12.313){} node[anchor=north west, align=left, text width=1.327cm, inner sep=6pt] at (14.5, -11.875){Ch2};
		\draw[-latex] (15, -12.5) -- (15, -13);
		\node[shape=rectangle, minimum width=0.465cm, minimum height=0.465cm](N1) at (12.74, -14.01){} node[anchor=center] at (N1.text){$V_s$} node[anchor=north west, align=left, text width=0.077cm, inner sep=6pt] at (12.49, -13.76){};
	\end{tikzpicture}
        \caption{Generador conectado a L2}
        \label{fig:circp3.2}
    \end{minipage}
\end{figure}
	

\subsection{Datos recolectados}

	\subsubsection*{Parametros del inductor}
	
	En primer lugar se obtuvieron los siguientes valores de inductancia a 50kHz con el medidor LRC.
	\begin{itemize}
	\item $L_1$ = 0,135 $mH$
	\item $L_2$ = 1,931 $mH$
	\end{itemize}
	
	Luego, con el generador de ondas configurado con una onda senoidal de 3 $V_{pp}$ a 50kHz con offset nulo, se midieron los siguientes valores (Observar tabla \ref{tab:mediciones_tensiones_bobinas})
	\begin{table}[h!]
	\centering
	\renewcommand{\arraystretch}{1.2} % Espaciado vertical más cómodo
	\setlength{\tabcolsep}{10pt}      % Espaciado horizontal
	\begin{tabular}{|l|l|l|}
	\hline
	\textbf{} & $V_{L1} [V] $& $V_{L2} [V] $ \\ \hline

	Generador conectado a L1
	& $2{,}10~\text\, \angle\, 0^\circ$ 
	& $4{,}34~\text\, \angle\, 182^\circ$ \\ \hline

	Generador conectado a L2
	& $3{,}10~\text\, \angle\, 0^\circ$
	& $0{,}45~\text\, \angle\, 181^\circ$ \\ \hline

	\end{tabular}
	\caption{Tensiones pico a pico medidas en cada bobina con sus respectivas fases.}
	\label{tab:mediciones_tensiones_bobinas}
	\end{table}

 \subsubsection*{Fuente de tensión con offset}
 
 	En la segunda parte al analizar variar el offset, se obtuvieron las siguientes mediciones variando el offset a la señal usada en el apartado anterior:

\begin{center}

% --- Fila superior (2 imágenes) ---
\begin{minipage}[b]{0.45\textwidth}
  \centering
  \includegraphics[width=\textwidth]{tp6_part2_1.png}
  \captionof{figure}{Fuente con offset de -5 V.}
  \label{fig:off1}
\end{minipage}
\hfill
\begin{minipage}[b]{0.45\textwidth}
  \centering
  \includegraphics[width=\textwidth]{tp6_part2_3.png}
  \captionof{figure}{Fuente con offset de -2,5 V.}
  \label{fig:off2}
\end{minipage}

\vspace{1.5em}

% --- Imagen central ---
\begin{minipage}[b]{0.5\textwidth}
  \centering
  \includegraphics[width=\textwidth]{tp6_part2_4.png}
  \captionof{figure}{Fuente con offset de 0 V.}
  \label{fig:off3}
\end{minipage}

\vspace{1.5em}

% --- Fila inferior (2 imágenes) ---
\begin{minipage}[b]{0.45\textwidth}
  \centering
  \includegraphics[width=\textwidth]{tp6_part2_5.png}
  \captionof{figure}{Fuente con offset de 2,5 V.}
  \label{fig:off4}
\end{minipage}
\hfill
\begin{minipage}[b]{0.45\textwidth}
  \centering
  \includegraphics[width=\textwidth]{tp6_part2_6.png}
  \captionof{figure}{Fuente con offset de 5 V.}
  \label{fig:off5}
\end{minipage}

\end{center}

	En las imagenes se puede observar las dos señales (entrada y salida). La señal marrón es la señal capturada en la entrada del transformador. La señal verde es la salida del transformador.
	
\subsubsection*{Impedancia de entrada con carga}

	En esta sección se realizaron estas dos mediciones.
	
	\begin{center}
	
		\begin{minipage}[b]{0.45\textwidth}
  		\centering
  		\includegraphics[width=\textwidth]{tp6_part3_l1_1.png}
  		\captionof{figure}{Caída de tensión medida sobre $R_m$. Conexión lado elevador}
  		\label{fig:Corriente_elevador}
		\end{minipage}
		\hfill
		\begin{minipage}[b]{0.45\textwidth}
  		\centering
  		\includegraphics[width=\textwidth]{tp6_maqt2_l3.png}
  		\captionof{figure}{Caída de tensión medida sobre $R_m$. Conexión lado reductor}
  		\label{fig:Corriente_reductor}
		\end{minipage}
		\vspace{1.5em}
	
	\end{center}
	
	Tanto en la figura \ref{fig:Corriente_elevador} como en la \ref{fig:Corriente_reductor}, el Ch1 (marrón) es la señal de la fuente, Ch2 (verde oscuro) es la señal a la salida de la resistencia $R_m$ y M es diferencia de tensión entre Ch1 y Ch2.
	
	En el caso del generador conectado al lado elevador de tensión (L1), se obtuvieron los siguientes valores de tensión (Observar figura \ref{fig:Corriente_elevador}):
	
	\begin{itemize}
		\item Caída de tensión sobre $ R_m$ : $1,398 V_{pp} \angle -42^\circ $ 
	\end{itemize}

	En el caso del generador conectado al lado reductor de tensión (L2), se obtuvieron los siguientes valores de tensión (Observar figura \ref{fig:Corriente_reductor}):
	
	\begin{itemize}
		\item Caída de tensión sobre $ R_m$ : $241,4 mV_{pp} \angle 272^\circ $ 
	\end{itemize}


\subsection{Cálculos}

\subsubsection*{Parametros del transformador}

	Para la obtención de M, se partió del sistema de ecuaciones \ref{eq:sist_eq} y se despejaron las siguientes ecuaciones, considerando que $I_2$ = 0:
	
\begin{equation*}
V_{in} - R_{L in} \cdot I_{in} - j \omega L_{in} I_{in} = 0
\end{equation*}

\begin{equation*}
V_{out} = j \cdot \omega \cdot M \cdot I_{in} 
\end{equation*}

	Al despejar y hacer el cociente de ambas, se puede llegar a:
	
\begin{equation*}
	\frac{V_{out}}{V_{in}} = \frac{j \, \omega \, M}{R_{L in} + j \, \omega \, L_{in}}
	\label{eq:acoplamiento_mutuo}
\end{equation*}

	Finalmente, como la resistencia del inductor es muy chica a comparación de $j\cdot \omega$ (con f = 50 kHz), se desprecia R llegando a la ecuación final:
	
\begin{equation}
    \frac{V_{out}}{V_{in}} \cdot L_{in} = M
    \label{eq:acoplamiento_mutuo}
\end{equation}

con $L_{in}$ el valor del bobinado conectado al generador, $V_{in}$ la tensión en la entrada del transformador y $V_{out}$ la tensión de la salida.

A partir de esta última ecuación se puede conseguir, mediante la fase, el signo de M, y al tomar el modulo, se puede despejar K mediante la ecuación \ref{eq:coef_acop}.

Con esto en cuenta se llegaron a los siguientes resultados:

\begin{itemize}
\item M = $0,279 mH \angle -178^\circ$ $\approx$ -0,279 mH
\item K = 0,546 
\end{itemize}
A partir de conectar $V_l1$ al generador.

\begin{itemize}
\item M = $0,28 mH \angle -179^\circ$ $\approx$ -0,28 mH
\item K = 0,549 
\end{itemize}
A partir de conectar $V_{L2}$ al generador.

Otro valor que se trabajo fue el factor de relación de vueltas para un transformador ideal. Aquí se calculó el n de varias formas:

\begin{itemize}
\item Mediante la ecuación \ref{eq:trafo_v} con $V_1$ la entrada de $L_1$ : n = 3,78
\item Mediante la ecuacion \ref{eq:trafo_v} con $V_2$ la entrada de $L_2$: $\frac{1}{n}$ = 0,2658 $\implies$ n = 6,89
\item Mediante la ecuación \ref{eq:rel_vuelta} : n= 3,78
\end{itemize}

\subsubsection*{Impedancia de entrada con carga}

	Mediante la ley de Ohm se consiguieron los siguientes valores de impedancia:
	
	\begin{itemize}
\item Con $R_L$ conectada a L2: $I_s = 14{,}87\text{ mA}\,\angle -42^\circ \implies Z_s = 101{,}35\,\Omega\,\angle 42^\circ$
\item Con $R_L$ conectada a L1: $I_s = 2{,}568\text{ mA}\,\angle -88^\circ \implies Z_s = 584{,}112\,\Omega\,\angle 88^\circ$
\end{itemize}