Los errores relativos observados en los parámetros obtenidos por conversión desde $Z$ son
significativos en varios casos, especialmente en las componentes de transferencia ($Y_{21}$, $T_{12}$).
Esto sugiere que las mediciones directas ofrecen una representación más precisa del comportamiento de
los cuadripolos en condiciones reales, y que los modelos de conversión deben ser revisados o ajustados
para mejorar su fidelidad.
\par En el caso de los cuadripolos conectados entre sí, se observaron buenos resultados (aún así con error bastante grande) para los 
conexionados en serie y en paralelo, pero no para el conexionado en cascada. Se considera que esto 
probablemente se deba al error humano en las mediciones, entre otras cosas por la la continua conexión 
y desconexión de componentes y la falta de experiencia del grupo de laboratorio.