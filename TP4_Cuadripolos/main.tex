\documentclass{article}

%%%%%%%%%%%%%%% LIBRERIAS %%%%%%%%%%%%%%%%%%%%%
\usepackage{amsmath}
\usepackage{caption}
\usepackage{physics}
\usepackage{titlesec}
\usepackage{titletoc}
\usepackage{graphicx}
\usepackage[spanish,es-tabla]{babel} % 'es-tabla' cambia Cuadro→Tabla
\usepackage{hyperref}                % cargar después de babel
\usepackage{float}
\usepackage{circuitikz}
\usepackage[left=3cm,right=3cm]{geometry} %para margenes
\usepackage{amsmath}
\usepackage{subfiles} 
%%%%%%%%%%%%%%%%%%% VARIABLES %%%%%%%%%%%%%%%%%%%%
\newcommand{\Facultad}{Instituto Tecnológico \\de\\ Buenos Aires} %constantes
\newcommand{\TPn}{Trabajo Práctico N° 4}
\newcommand{\TPtema}{Respuesta Transitoria}
\renewcommand{\thesection}{\arabic{section}}          % 2
\renewcommand{\thesubsection}{\quad \alph{subsection}}   % a
\renewcommand{\thesubsubsection}{\quad \thesubsection.~\roman{subsubsection}} % a. i
\graphicspath{{imagenesFiguras/}} %para que acceda a las fotos en la carpeta directamente


%%%%%%%%%%%%%%%%%% FORMATO TÍTULO Y NUMERACIÓN %%%%%%%%%%%%%%%%%%%
% Numeración de secciones
\renewcommand{\thesection}{\arabic{section}.}          
\renewcommand{\thesubsection}{\thesection\arabic{subsection}}       
\renewcommand{\thesubsubsection}{$\alph{subsubsection})$}

% Numerar hasta subsubsecciones
\setcounter{secnumdepth}{3}

% Formato de títulos
\titleformat{\section}{\Huge\bfseries}{\thesection}{1em}{}
\titleformat{\subsection}{\LARGE\bfseries}{\thesubsection}{0.5em}{}
\titleformat{\subsubsection}{\large\bfseries}{\thesubsubsection}{0.5em}{}



%%%%%%%%%%%%%%%%%%% ARCHIVO %%%%%%%%%%%%%%%%%%%%%%%%
\begin{document}

%%%CARATULA%%%
\begin{titlepage} %creo portada

        \begin{flushleft}
            \centering
            \includegraphics[width=0.3\textwidth]{Logo_ITBA.png}
        \end{flushleft}

        \centering
            
        {\scshape\LARGE \Facultad \par} %\par sirve para indicar un final de parrafo
        \vspace{1cm}                    %esto hace un espacio entre lineas de 1cm


        {\huge\bfseries \TPn \par}
        \vspace{1.5cm}
        {\Large Teoría de Circuitos I\\ 25.10 \par}
        \vfill                      %sirve para rellenar el espacio y quede simétrico. Si se añaden otros, se dividen el espacio de forma equitativa
        {\Large \bfseries Grupo N° 2 \par}
        \vspace{1cm}
        {\large Juan Bautista Correa Uranga \hfill Legajo: 65016 \par} %\hfill sirve para hacerlo simétrico
        {\large Juan Ignacio Caorsi \hfill Legajo: 65532  \par}
        {\large Rita Moschini \hfill Legajo: 67026 \par} 
        \vfill
        {\large \today\par}
        \vfill

\end{titlepage}

 %%%RESUMEN%%%
{\centering \LARGE \bfseries Resumen \par}


\newpage

%%%INDICE%%%
\tableofcontents %esto sirve para crear el índice
\newpage

%%%Introduccion%%%
\section{Introducción}
    \subfile{introduccion}   

\section{Desarrollo}

    \subfile{desarrolloSinAnalisis}
    \subfile{desarrolloAnalisis}
        
\section{Conclusiones}
    \subfile{conclusiones}
	

\end{document}
