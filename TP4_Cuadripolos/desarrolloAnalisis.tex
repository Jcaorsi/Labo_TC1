\subsection{Análisis}

Al comparar los datos calculados, se observaron grandes diferencias en los valores. Aun así se pudieron observar los siguientes rasgos en cada cuadripolo.

\subsubsection{Cuadripolos individuales}


Una de las primeras observaciones que se pueden hacer para el cuadripolo 9603 a partir de la tabla \ref{tab:matriz_Z9603}, es que todos los valores imaginarios de los parámetros son negativos. De aquí se puede inducir que el mismo cuenta con componentes predominantemente capacitivos y no inductivos. Otra conclusión que se puede obtener de esta tabla, es que este cuadripolo no cumple con las condiciones de simetría ni reciprocidad. Esto puesto a que $Z_{11} \neq Z_{22}$, por lo tanto no es simétrico; $Z_{12} \neq Z_{21}$ por lo tanto no es recíproco.
Al repetir el procedimiento que se hizo en el cuadripolo anterior, al observar la matriz Z (tabla \ref{tab:matriz_Z9609}), también se puede deducir que el cuadripolo tiene componentes capacitivos, puesto a que la parte imaginaria de cada parametro es negativa. También se puede observar que este cuadripolo no es ni simétrico ni recíproco. 

Se comparan los valores medidos y los obtenidos por conversión para los parámetros $Y$ y $T$. Al comparar las tablas \ref{tab:matriz_Y9603} (medido)y \ref{tab:matriz_Y9603_indirecta} (teórico), se observa una discrepancia significativa en los valores fuera de la diagonal, lo que sugiere que el modelo de conversión no refleja adecuadamente el acoplamiento observado en la medición. El error relativo en la componente $Y_{21}$ es:

%\subsubsection*{Parámetros $Y$}

%La matriz medida fue:


%\[
%Y_{medido} =
%\begin{bmatrix}
%10 & -7.9 - 0.55j \\
%-9.34 & 12.79 + 5.69j
%\end{bmatrix} \,\mathrm{mS}
%\]



%La matriz obtenida por conversión desde $Z$ fue:


%\[
%Y_{\mathrm{teorico}} =
%\begin{bmatrix}
%14.55 - 1.57j & -12.16 + 0.89j \\
%-14.76 + 1.07j & 17.277 + 5.23j
%\end{bmatrix} \,\mathrm{mS}
%\]



%Se observa una discrepancia significativa en los valores fuera de la diagonal, lo que sugiere que el modelo de conversión no refleja adecuadamente el acoplamiento observado en la medición. El error relativo en la %componente $Y_{21}$ es: (esto lo muevo arriba)



\[
\varepsilon_{rel} = \frac{|\text{teórico} - \text{medido}|}{|\text{medido}|} = \frac{|(-14.76 + 1.07j) + 9.34|}{|-9.34|} \approx 0,59
\] 

 (ya se corrigió, pasamos de 1,68 a 0,59)



Este error relativo del 59\% indica una diferencia importante. Este mismo patrón se cumple con las otras componentes de las dos matrices. Algo parecido surge con el parámetro T. Al repetir la cuenta anterior pero comparando las tablas \ref{tab:matriz_T9603} (medido) y \ref{tab:matriz_T9603_indirecta} (teórico), se llega a un error relativo de 24\% para el parámetro $T_{11}$ y un error del 59\% para el parametro $T_{12}$ donde se nota una mayor diferencia al comparar un valor real con uno imaginario. \par

Al pasar a evaluar las diferencias en el cuadripolo 9609, se encuentran las mismas diferencias entre los valores. Al observar las matrices \ref{tab:matriz_Y_9609} (medido) y \ref{tab:matriz_Y9609_indirecta} (teorica), en el parametro $Y_{21}$ se puede osbervar un error del 265\%. Algo muy parecido pasa con los otros parametros.

Este gran error se puede explicar por varios motivos. El más importante es el error humano, puesto a que en esta experiencia se tuvieron que realizar muchas mediciones repetidas cambiando el conexionado y usando varias funciones del osciloscopio. Sumado a esto, nuestro grupo no contaba con una gran experiencia en el manejo del osciloscopio, lo que dificultó la toma de mediciones. Otras causas consideradas fueron el ruido en las mediciones causado por el conexionado.

%\subsubsection*{Parámetros $T$}

%Medido:


%\[
%T_{medido} =
%\begin{bmatrix}
%1.47 + 0.65j & 107.04\,\Omega \\
%6.31 + 7.53j\,\mathrm{mS} & 1.07
%\end{bmatrix}
%\]



%Convertido desde $Z$:


%\[
%T_{teorico} =
%\begin{bmatrix}
%1.14 + 0.44j & 67.37 + 4.91j\,\Omega \\
%5.09 + 5.45j\,\mathrm{mS} & 0.99 - 0.03j
%\end{bmatrix}
%\]


%El error relativo en la componente $T_{12}$ es especialmente significativo:

%\[
%\varepsilon_{rel} = \frac{|(67.37 + 4.91j) - (-107.04)|}{|107.04|} \approx 1.63 (cambiar)
%\]



%Esto sugiere una discrepancia de más del 160\% en la impedancia de transferencia.

%\subsubsection{Cuadripolo 9609}

%Al repetir el procedimiento que se hizo en el cuadripolo anterior, al observar la matriz Z (tabla \ref{tab:matriz_Z9609}), también se puede deducir que el cuadripolo tiene componentes capacitivos, puesto a que la parte imaginaria de cada parametro es negativa. También se puede observar que este cuadripolo no es ni simétrico ni reciproco. 

%\subsubsection*{Parámetros $Y$}

%Medido:


%\[
%Y_{medido} =
%\begin{bmatrix}
%4.18 + 1.52j & - 2.63 + 0.32j \\
%- 1.95 + 0.60j & 2.91 + 0.20j
%\end{bmatrix} \,\mathrm{mS}
%\]



%Convertido desde $Z$:


%\[
%Y_{teorico} =
%\begin{bmatrix}
%7.98 + 0.47j & -2.85 + 2.19j \\
%-7.29 - 0.30j & 5.12 - 0.96j
%\end{bmatrix} \,\mathrm{mS}
%\]



%El error relativo en $Y_{21}$ es:



%\[
%\varepsilon_{rel} = \frac{|(-7.29 - 0.30j) - 1.95 + 0.60j|}{|1.95 - 0.60j|} \approx 4.3 (cambiar)
%\]



%Lo que representa un error del 430\%, indicando una inversión de signo y magnitud.

%\subsubsection*{Parámetros $T$}

%Medido:


%\[
%T_{medido} =. (cambiar ) tabla 13
%\begin{bmatrix}
%1.324 + 0.704j & -468.21 - 143.15j\,\Omega \\
%2.7 + 5.8j\,\mathrm{mS} & -1.74 - 1.31j
%\end{bmatrix}
%\]



%Convertido desde $Z$:


%\[
%T_{teorico} = (tabla 16) (cambiar)
%\begin{bmatrix}
%0.69 - 0.16j & 136.93 - 5.69j\,\Omega \\
%2.78 + 1.24j\,\mathrm{mS} & 1.10 + 0.02j
%\end{bmatrix}
%\]



%El error relativo en $T_{12}$ es:



\[
\varepsilon_{rel} = \frac{|(136.93 - 5.69j) - (-468.21 - 143.15j)|}{|468.21 + 143.15j|} \approx 1.1 (cambiar)
\]



Un error del 110\% que indica una discrepancia importante en la impedancia de transferencia.

\subsubsection{Parametros combinados}
Como se puede observar en la tabla \ref{medicionesTestBrune} de la medicion del test de Brune en el conexionado en serie se obtuvo un resultado de $1,6 V$, lejano a $0 V$, por lo que resulta esperable que la suma de las matrices de impedancia de un valor distinto de la matriz impedancia medida de los cuadripolos conectados en serie. Luego, con el conexionado en paralelo, el test de Brune da $0 V$, por lo que es correcto afirmar que la suma de las matrices admitancia de ambos cuadripolos deberia resultar en la matriz admitancia del conjunto. Sin embargo, esto no se refleja en los calculos. El error relativo en el coeficiente $Y_{21}$ es

\begin{equation*}
    \varepsilon_r = \frac{\left| (8,48 - 0,59j) - (11,30 - 0,60j) \right|} {\left| 8,48 - 0,59j \right|} \approx 0,33
\end{equation*}

esto es, los parametros admitancia del conexionado en paralelo discrepan de los esperados en un 33%.
