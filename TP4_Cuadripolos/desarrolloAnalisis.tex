\subsection{Análisis}

Al comparar los datos calculados, se observaron grandes diferencias en los valores. 

\subsubsection{Cuadripolo 9603}


Una de las primeras observaciones que se pueden hacer de la tabla \ref{tab:matriz_Z9603}, es que todos los valores imaginarios de los parametros, son negativos. De aquí se puede inducir que el mismo cuenta con componentes predominantemente capacitivos y no inductivos. Otra conclusión que se puede obtener de esta tabla, es que este cuadripolo no cumple con las condiciones de simetría ni reciprocidad.

Se comparan los valores medidos y los obtenidos por conversión para los parámetros $Y$ y $T$.

\subsubsection*{Parámetros $Y$}

La matriz medida fue:


\[
Y_{medido} =
\begin{bmatrix}
10 & 7.9 + 0.55j \\
9.34 & 12.79 + 5.69j
\end{bmatrix} \,\mathrm{mS}
\]



La matriz obtenida por conversión desde $Z$ fue:


\[
Y_{\mathrm{teorico}} =
\begin{bmatrix}
14.55 - 1.57j & -12.16 + 0.89j \\
-14.76 + 1.07j & 17.277 + 5.23j
\end{bmatrix} \,\mathrm{mS}
\]



Se observa una discrepancia significativa en los valores fuera de la diagonal, lo que sugiere que el modelo de conversión no refleja adecuadamente el acoplamiento observado en la medición. El error relativo en la componente $Y_{21}$ es:



\[
\varepsilon_{rel} = \frac{|\text{teórico} - \text{medido}|}{|\text{medido}|} = \frac{|(-14.76 + 1.07j) - 9.34|}{|9.34|} \approx 1.68
\]



Este error relativo del 168\% indica una diferencia importante.

\subsubsection*{Parámetros $T$}

Medido:


\[
T_{medido} =
\begin{bmatrix}
1.47 + 0.65j & -107.04\,\Omega \\
6.31 + 7.53j\,\mathrm{mS} & -1.07
\end{bmatrix}
\]



Convertido desde $Z$:


\[
T_{teorico} =
\begin{bmatrix}
1.14 + 0.44j & 67.37 + 4.91j\,\Omega \\
5.09 + 5.45j\,\mathrm{mS} & 0.99 - 0.03j
\end{bmatrix}
\]



El error relativo en la componente $T_{12}$ es especialmente significativo:



\[
\varepsilon_{rel} = \frac{|(67.37 + 4.91j) - (-107.04)|}{|107.04|} \approx 1.63
\]



Esto sugiere una discrepancia de más del 160\% en la impedancia de transferencia.

\subsubsection{Cuadripolo 9609}

Al repetir el procedimiento que se hizo en el cuadripolo anterior, al observar la matriz Z (tabla \ref{tab:matriz_Z9609}), también se puede deducir que el cuadripolo tiene componentes capacitivos, puesto a que la parte imaginaria de cada parametro es negativa. También se puede observar que este cuadripolo no es ni simétrico ni reciproco. 

\subsubsection*{Parámetros $Y$}

Medido:


\[
Y_{medido} =
\begin{bmatrix}
4.18 + 1.52j & 2.63 - 0.32j \\
1.95 - 0.60j & 2.91 + 0.20j
\end{bmatrix} \,\mathrm{mS}
\]



Convertido desde $Z$:


\[
Y_{teorico} =
\begin{bmatrix}
7.98 + 0.47j & -2.85 + 2.19j \\
-7.29 - 0.30j & 5.12 - 0.96j
\end{bmatrix} \,\mathrm{mS}
\]



El error relativo en $Y_{21}$ es:



\[
\varepsilon_{rel} = \frac{|(-7.29 - 0.30j) - 1.95 + 0.60j|}{|1.95 - 0.60j|} \approx 4.3
\]



Lo que representa un error del 430\%, indicando una inversión de signo y magnitud.

\subsubsection*{Parámetros $T$}

Medido:


\[
T_{medido} =
\begin{bmatrix}
1.324 + 0.704j & -468.21 - 143.15j\,\Omega \\
2.7 + 5.8j\,\mathrm{mS} & -1.74 - 1.31j
\end{bmatrix}
\]



Convertido desde $Z$:


\[
T_{teorico} =
\begin{bmatrix}
0.69 - 0.16j & 136.93 - 5.69j\,\Omega \\
2.78 + 1.24j\,\mathrm{mS} & 1.10 + 0.02j
\end{bmatrix}
\]



El error relativo en $T_{12}$ es:



\[
\varepsilon_{rel} = \frac{|(136.93 - 5.69j) - (-468.21 - 143.15j)|}{|468.21 + 143.15j|} \approx 1.1
\]



Un error del 110\% que indica una discrepancia importante en la impedancia de transferencia.

\subsection*{Conclusión}

Los errores relativos observados en los parámetros indirectos obtenidos por conversión desde $Z$ son significativos en varios casos, especialmente en las componentes de transferencia ($Y_{21}$, $T_{12}$). Esto sugiere que las mediciones directas ofrecen una representación más precisa del comportamiento de los cuadripolos en condiciones reales, y que los modelos de conversión deben ser revisados o ajustados para mejorar su fidelidad.

