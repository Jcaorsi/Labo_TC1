Este trabajo práctico aborda la descripción de redes de dos puertos o cuadripolos mediante sus distintos
parámetros. Se buscó adquirir experiencia en la obtención de los mismos,
verificar las ecuaciones de conversión, y predecir los parámetros de las redes de dos puertos resultantes
de conectar de distintas maneras dos cuadripolos con parámetros obtenidos previamente.

\subsection{Instrumental}
        En esta experiencia se utilizaron los siguientes instrumentos:
\begin{itemize}
  \item \href{https://www.rftesolutions.com/index.php?main_page=product_info&products_id=729}{Osciloscopio Keysight (Agilent) DSO6014A}
  \item \href{https://www.keysight.com/us/en/product/EDU33212A/waveform-generator-20mhz-2-channel.html}{Generador de ondas}  con 
        resistencia interna de $ 50 \Omega $ % lo tuvimos en cuenta?? vale la pena mencionarlo??
  \item Cuadripolo 9603
  \item Cuadripolo 9609
  \item Resistencia de $ 4,7 \Omega $
  \item Resistencia de $ 1 K\Omega $
\end{itemize}
Para ambos cuadripolos, la tensión máxima de entrada es de 15 V, y la corriente máxima de entrada vale 50 mA.

\subsection{Marco teórico}
Muchos circuitos prácticos tienen solamente dos puertos de acceso, es decir, dos lugares donde las
señales pueden entrar o salir. En particular, una red de cuatro terminales se denomina red de dos
puertos cuando, para ambos pares de terminales, la corriente entrante a una terminal del par sale
por la otra terminal del par. \par

\begin{figure}[H]
    \centering
            \begin{tikzpicture}
            \node[shape=rectangle, draw, line width=1pt, minimum width=1.965cm, minimum height=1.965cm] at (5, 5){};
            \draw (4.25, 5.5) -- (4.25, 5.75);
            \draw (4.5, 5.5) -- (4.5, 5.75);
            \draw (4.75, 5.75) -- (4.75, 5.5);
            \draw (4, 5.75) -- (2.5, 5.75);
            \draw (4, 4.25) -- (2.5, 4.25);
            \draw (6, 5.75) -- (7.5, 5.75);
            \draw (6, 4.25) -- (7.5, 4.25);
            \draw[-latex] (2.5, 6) -- (3.25, 6);
            \draw[-latex] (3.25, 4.5) -- (2.5, 4.5);
            \draw[-latex] (7.5, 6) -- (6.75, 6);
            \draw[-latex] (6.75, 4.5) -- (7.5, 4.5);
            \node[plain crossing] at (1.89, 5.75){};
            \node[plain crossing] at (8.14, 5.75){};
            \draw (2, 4.5) -- (1.75, 4.5);
            \draw (8, 4.5) -- (8.25, 4.5);
            \node[shape=rectangle, minimum width=0.465cm, minimum height=0.465cm] at (2.5, 6.25){} node[anchor=center, align=center, text width=0.077cm, inner sep=6pt] at (2.5, 6.25){\scriptsize I1};
            \node[shape=rectangle, minimum width=0.465cm, minimum height=0.465cm] at (2.5, 4.75){} node[anchor=center, align=center, text width=0.077cm, inner sep=6pt] at (2.5, 4.75){\scriptsize I1};
            \node[shape=rectangle, minimum width=0.465cm, minimum height=0.465cm] at (7.5, 6.25){} node[anchor=center, align=center, text width=0.077cm, inner sep=6pt] at (7.5, 6.25){\scriptsize I2};
            \node[shape=rectangle, minimum width=0.465cm, minimum height=0.465cm] at (7.5, 4.75){} node[anchor=center, align=center, text width=0.077cm, inner sep=6pt] at (7.5, 4.75){\scriptsize I2};
            \node[shape=rectangle, minimum width=0.465cm, minimum height=0.465cm] at (1.85, 5.144){} node[anchor=center, align=center, text width=0.077cm, inner sep=6pt] at (1.85, 5.144){\scriptsize V1};
            \node[shape=rectangle, minimum width=0.465cm, minimum height=0.465cm] at (8.25, 5.117){} node[anchor=center, align=center, text width=0.077cm, inner sep=6pt] at (8.25, 5.117){\scriptsize V2};
        \end{tikzpicture}
\caption{Red de dos puertos común.}
    \label{fig: cuadripolo}
\end{figure}

Con el fin de describir este tipo de redes sin conocer o profundizar sobre su composición interna,
es útil conocer las relaciones entre los voltajes y las corrientes de los puertos.
Para eso, se definen los parámetros impedancia ($Z$), admitancia ($Y$) y transmisión ($T$) de la siguiente manera:

\begin{equation*}
    Z  =
    \begin{pmatrix}
        Z_{11} & Z_{12} \\
        Z_{21} & Z_{22}
    \end{pmatrix}
    \qquad Y = 
    \begin{pmatrix}
        Y_{11} & Y_{12} \\
        Y_{21} & Y_{22}
    \end{pmatrix}
    \qquad T = 
    \begin{pmatrix}
        A & -B \\
        C & -D
    \end{pmatrix}
\end{equation*}
tal que
\begin{equation*}
    \begin{cases}
        V_1 = I_1 \cdot Z_{11} + I_2 \cdot Z_{12} \\ 
        V_2 = I_1 \cdot Z_{21} + I_2 \cdot Z_{22}
    \end{cases}
    \qquad \qquad
    \begin{cases}
        I_1 = V_1 \cdot Y_{11} + V_2 \cdot Y_{12} \\ 
        I_2 = V_1 \cdot Y_{21} + V_2 \cdot Y_{22}
    \end{cases}
    \qquad \qquad
    \begin{cases}
        V_1 = V_2 \cdot A + I_1 \cdot (-B) \\ 
        I_1 = V_2 \cdot C + I_1 \cdot (-D)
    \end{cases}
\end{equation*}

Si por ejemplo se quisiera obtener el parámetro $ Z_{11} $, anulando $ I_2$ y despejando en la ecuación $ V_1 = I_1 \cdot Z_{11} + I_2 \cdot Z_{12} $ se tiene
\begin{equation*}
    Z_{11} = \frac{V_1}{I_1} \Biggr|_{\substack{I_2=0}} % Aunque esté rojo, no tira error!
\end{equation*}

\par Otro aspecto que se buscó estudiar es si un cuadripolo era simétrico o recíproco. Se dice que un cuadripolo es \textbf{simétrico} si que tiene la misma
estructura vista desde cualquiera de sus dos puertos, esto es si se cumplen las siguientes ecuaciones en simultáneo:
\begin{equation*}
    Z_{11}=Z_{22} \quad \wedge \quad Y_{11}=Y_{22} \quad \wedge \quad A=D
\end{equation*}
Luego, un cuadripolo es \textbf{recíproco} si es simétrico y solo tiene elementos pasivos (resistencias, inductores o capacitores, no 
fuentes). En estos casos, se puede afirmar
\begin{equation*}
    Z_{12}=Z_{21} \quad \wedge \quad Y_{12}=Y_{21} \quad \wedge \quad AD-BE=1
\end{equation*}

\par Por otro lado, la cátedra provee una tabla con ecuaciones que permiten obtener todos los parámetros a partir de cualquier otro. Los que serán 
utilizados en esta práctica son los siguientes:
\begin{equation*}
    Y =
    \begin{pmatrix}
        \frac{Z_{22}}{\Delta z} & \frac{-Z_{12}}{\Delta z} \\ \\
        \frac{-Z_{21}}{\Delta z} & \frac{-Z_{11}}{\Delta z}
    \end{pmatrix}
    \qquad \qquad T =
    \begin{pmatrix}
        \frac{Z_{11}}{Z_{21}} & \frac{\Delta z}{Z_{21}} \\ \\
        \frac{1}{Z_{21}} & \frac{Z_{22}}{Z_{21}}
    \end{pmatrix}
\end{equation*}

\par Por último, se estudió la relación entre parámetros para los distintos conexionados entre cuadripolos que se presentan a continuación


\subsubsection*{Conexión Serie}

\begin{figure}[H]
    \begin{equation*}
        \begin{tikzpicture}
            \node[shape=rectangle, draw, line width=1pt, minimum width=1.965cm, minimum height=1.965cm] at (5, 5){};
            \draw (4.25, 5.5) -- (4.25, 5.75);
            \draw (4.5, 5.5) -- (4.5, 5.75);
            \draw (4.75, 5.75) -- (4.75, 5.5);
            \draw (4, 5.75) -- (2.5, 5.75);
            \draw (4, 1.75) -- (2.5, 1.75);
            \draw (6, 5.75) -- (7.5, 5.75);
            \draw (6.011, 1.761) -- (7.511, 1.761);
            \draw[-latex] (2.5, 6) -- (3.25, 6);
            \draw[-latex] (3.25, 2) -- (2.5, 2);
            \draw[-latex] (7.5, 6) -- (6.75, 6);
            \draw[-latex] (6.75, 2.022) -- (7.5, 2.022);
            \node[plain crossing] at (1.879, 5.75){};
            \node[plain crossing] at (8.14, 5.75){};
            \node[shape=rectangle, minimum width=0.465cm, minimum height=0.465cm] at (2.5, 6.25){} node[anchor=center, align=center, text width=0.077cm, inner sep=6pt] at (2.5, 6.25){\scriptsize I1};
            \node[shape=rectangle, minimum width=0.465cm, minimum height=0.465cm] at (2.5, 2.25){} node[anchor=center, align=center, text width=0.077cm, inner sep=6pt] at (2.5, 2.25){\scriptsize I1};
            \node[shape=rectangle, minimum width=0.465cm, minimum height=0.465cm] at (7.5, 6.25){} node[anchor=center, align=center, text width=0.077cm, inner sep=6pt] at (7.5, 6.25){\scriptsize I2};
            \node[shape=rectangle, minimum width=0.465cm, minimum height=0.465cm] at (7.5, 2.25){} node[anchor=center, align=center, text width=0.077cm, inner sep=6pt] at (7.5, 2.25){\scriptsize I2};
            \node[shape=rectangle, minimum width=0.465cm, minimum height=0.465cm] at (1.872, 4){} node[anchor=center, align=center, text width=0.077cm, inner sep=6pt] at (1.872, 4){\scriptsize V1};
            \node[shape=rectangle, minimum width=0.465cm, minimum height=0.465cm] at (8.111, 4){} node[anchor=center, align=center, text width=0.077cm, inner sep=6pt] at (8.111, 4){\scriptsize V2};
            \node[shape=rectangle, draw, line width=1pt, minimum width=1.965cm, minimum height=1.965cm] at (5, 2.5){};
            \draw (4.25, 3.25) -| (4.25, 3);
            \draw (4.5, 3.25) -| (4.5, 3);
            \draw (4.75, 3.25) -| (4.75, 3);
            \draw (4, 4.25) -- (3.5, 4.25) -| (3.5, 3.25) -- (4, 3.25);
            \draw (6, 4.25) -- (6.5, 4.25) -| (6.5, 3.25) -- (6, 3.25);
            \draw (2, 2) -- (1.75, 2);
            \draw (8, 2) -- (8.25, 2);
        \end{tikzpicture}
    \end{equation*}
\caption{Dos cuadripolos conectados en serie.}
    \label{fig: conexionSerie}
\end{figure}

En este conexionado, se cumple que la matriz impedancia $Z$ resultante es
\begin{equation*}
    Z = Z_A + Z_B
\end{equation*}
siendo $ Z_A $ y $ Z_B $ las matrices impedancia de los respectivos cuadripolos.

Sin embargo, esto no sucede siempre, puesto que en algunos casos, el conexionado de los cuadripolos entre sí modifica los parámetros 
originales. Para saber si esto sucede o no, se realiza el \textbf{Test de Brune}.

% La escala de la resistencia y de la fuente es 0,6 siempre

\begin{figure}[h]
\centering
\begin{minipage}{0.45\textwidth}
    \begin{equation*}
        \begin{tikzpicture}
            \node[shape=rectangle, draw, line width=1pt, minimum width=1.965cm, minimum height=1.965cm] at (5, 5){};
            \draw (4.25, 5.5) -- (4.25, 5.75);
            \draw (4.5, 5.5) -- (4.5, 5.75);
            \draw (4.75, 5.75) -- (4.75, 5.5);
            \draw (4, 5.75) -- (2.5, 5.75);
            \draw (4, 1.75) -- (2.5, 1.75);
            \draw (6, 5.75) -| (7.25, 5.75);
            \draw (6, 1.75) -- (7.25, 1.75);
            \node[shape=rectangle, draw, line width=1pt, minimum width=1.965cm, minimum height=1.965cm] at (5, 2.5){};
            \draw (4.25, 3.25) -| (4.25, 3);
            \draw (4.5, 3.25) -| (4.5, 3);
            \draw (4.75, 3.25) -| (4.75, 3);
            \draw (4, 4.25) -- (3.5, 4.25) -| (3.5, 3.25) -- (4, 3.25);
            \node[oscillator, xscale=0.6, yscale=0.6] at (2.794, 3.294){};
            \draw (7, 4.25) to[voltmeter, /tikz/circuitikz/bipoles/length=0.840cm] (7, 3.25);
            \draw (6, 4.25) -- (7, 4.25);
            \draw (6, 3.25) -- (7, 3.25);
            \draw (2.5, 5.75) to[american resistor, /tikz/circuitikz/bipoles/length=0.840cm] (2.5, 3.588);
            \draw (2.5, 3) -| (2.5, 1.75);
        \end{tikzpicture}
    \end{equation*}
\end{minipage}
\hspace{0.05\textwidth}% espacio entre figuras
\begin{minipage}{0.45\textwidth}
    \begin{equation*}
    \begin{tikzpicture}
        \node[shape=rectangle, draw, line width=1pt, minimum width=1.965cm, minimum height=1.965cm] at (5, 5){};
        \draw (4.25, 5.5) -- (4.25, 5.75);
        \draw (4.5, 5.5) -- (4.5, 5.75);
        \draw (4.75, 5.75) -- (4.75, 5.5);
        \draw (4, 5.75) -- (2.5, 5.75);
        \draw (4, 1.75) -- (2.5, 1.75);
        \draw (6, 5.75) -| (8, 5.75);
        \draw (6, 1.75) -| (8, 3.25);
        \node[shape=rectangle, draw, line width=1pt, minimum width=1.965cm, minimum height=1.965cm] at (5, 2.5){};
        \draw (4.25, 3.25) -| (4.25, 3);
        \draw (4.5, 3.25) -| (4.5, 3);
        \draw (4.75, 3.25) -| (4.75, 3);
        \node[oscillator, xscale=0.6, yscale=0.6] at (8.294, 3.544){};
        \draw (3.5, 4.25) to[voltmeter, /tikz/circuitikz/bipoles/length=0.840cm] (3.5, 3.25);
        \draw (6, 4.25) -- (7, 4.25);
        \draw (6, 3.25) -| (7, 4.25);
        \draw (8, 5.75) to[american resistor, /tikz/circuitikz/bipoles/length=0.840cm] (8, 3.838);
        \draw (3.5, 4.25) -- (4, 4.25);
        \draw (3.5, 3.25) -- (4, 3.25);
    \end{tikzpicture}    
\end{equation*}
\end{minipage}
\caption{Test de Brune para el conexionado en Serie.}
\label{fig: testBruneSerie}
\end{figure}


% No sé por qué, si pongo dos dibujos en una misma \begin{equation*} me tira error

Si se cumple que la tensión del voltímetro es cero o cercana a cero en ambos casos, entonces se cumple la condición de Brune y se puede 
obtener la matriz Z de la manera explicada.

\subsubsection*{Conexión en Paralelo}

\begin{figure}[H]
    \begin{equation*}
        \begin{tikzpicture}
            \node[shape=rectangle, draw, line width=1pt, minimum width=1.965cm, minimum height=1.965cm] at (5, 5){};
            \draw (4.25, 5.5) -- (4.25, 5.75);
            \draw (4.5, 5.5) -- (4.5, 5.75);
            \draw (4.75, 5.75) -- (4.75, 5.5);
            \draw (4, 5.75) -- (2.5, 5.75);
            \draw (4, 3) -- (2.5, 3);
            \draw (6, 5.75) -- (7.5, 5.75);
            \draw (6, 3) -- (7.5, 3);
            \draw[-latex] (2.5, 6) -- (3.25, 6);
            \draw[-latex] (3.25, 3.261) -- (2.5, 3.261);
            \draw[-latex] (7.5, 6) -- (6.75, 6);
            \draw[-latex] (6.739, 3.272) -- (7.489, 3.272);
            \node[plain crossing] at (1.879, 5.75){};
            \node[plain crossing] at (8.14, 5.75){};
            \node[shape=rectangle, minimum width=0.465cm, minimum height=0.465cm] at (2.5, 6.25){} node[anchor=center, align=center, text width=0.077cm, inner sep=6pt] at (2.5, 6.25){\scriptsize I1};
            \node[shape=rectangle, minimum width=0.465cm, minimum height=0.465cm] at (2.5, 3.5){} node[anchor=center, align=center, text width=0.077cm, inner sep=6pt] at (2.5, 3.5){\scriptsize I1};
            \node[shape=rectangle, minimum width=0.465cm, minimum height=0.465cm] at (7.5, 6.25){} node[anchor=center, align=center, text width=0.077cm, inner sep=6pt] at (7.5, 6.25){\scriptsize I2};
            \node[shape=rectangle, minimum width=0.465cm, minimum height=0.465cm] at (7.489, 3.489){} node[anchor=center, align=center, text width=0.077cm, inner sep=6pt] at (7.489, 3.489){\scriptsize I2};
            \node[shape=rectangle, minimum width=0.465cm, minimum height=0.465cm] at (1.75, 4.5){} node[anchor=center, align=center, text width=0.077cm, inner sep=6pt] at (1.75, 4.5){\scriptsize V1};
            \node[shape=rectangle, minimum width=0.465cm, minimum height=0.465cm] at (8.25, 4.5){} node[anchor=center, align=center, text width=0.077cm, inner sep=6pt] at (8.25, 4.5){\scriptsize V2};
            \node[shape=rectangle, draw, line width=1pt, minimum width=1.965cm, minimum height=1.965cm] at (5, 2.5){};
            \draw (4.25, 3.25) -| (4.25, 3);
            \draw (4.5, 3.25) -| (4.5, 3);
            \draw (4.75, 3.25) -| (4.75, 3);
            \draw (4, 4.25) -- (3.5, 4.25) -- (3.5, 1.75) -| (4, 3.25);
            \draw (6, 4.25) -- (6.5, 4.25) -- (6.5, 1.75) -| (6, 3.25);
            \draw (2, 3.25) -- (1.75, 3.25);
            \draw (7.989, 3.239) -- (8.239, 3.239);
        \end{tikzpicture}
    \end{equation*}
\caption{Dos cuadripolos conectados en paralelo.}
\label{fig: conexionParalelo}
\end{figure}

En este conexionado, se cumple que la matriz admitancia $Y$ resultante es
\begin{equation*}
    Y = Y_A + Y_B
\end{equation*}
siendo $ Y_A $ e $ Y_B $ las matrices admitancia de los respectivos cuadripolos.

De forma análoga al conexionado serie, no siempre se cumple esta relación. El esquema de conexionado del \textbf{Test de Brune} en este caso es

\begin{figure}[h]
\centering
\begin{minipage}{0.45\textwidth}
    \centering
    \begin{equation*}
        \begin{tikzpicture}
            \node[shape=rectangle, draw, line width=1pt, minimum width=1.965cm, minimum height=1.965cm] at (5.529, 1.971){};
            \draw (4.779, 2.721) -| (4.779, 2.471);
            \draw (5.029, 2.721) -| (5.029, 2.471);
            \draw (5.279, 2.721) -| (5.279, 2.471);
            \node[shape=rectangle, draw, line width=1pt, minimum width=1.965cm, minimum height=1.965cm] at (5.5, 4.5){};
            \draw (4.75, 5.25) -| (4.75, 5);
            \draw (5, 5.25) -| (5, 5);
            \draw (5.25, 5.25) -| (5.25, 5);
            \draw (4.5, 5.25) -- (4, 5.25) -| (4, 3.75) -- (4.5, 3.75);
            \draw (4.5, 2.75) -- (4, 2.75) -| (4, 1.25) -- (4.5, 1.25);
            \draw (6.5, 5.25) -- (8.5, 5.25);
            \draw (6.5, 1.25) -- (8.5, 1.25);
            \draw (6.5, 3.75) -- (7.75, 3.75) -- (7.75, 1.25);
            \draw (6.5, 2.75) -- (7, 2.75) -- (7, 5.25);
            \node[circ] at (7, 5.25){};
            \node[circ] at (7.75, 1.25){};
            \draw (8.5, 5.25) to[american resistor, /tikz/circuitikz/bipoles/length=0.840cm] (8.5, 3.088);
            \node[oscillator, xscale=0.6, yscale=0.6] at (8.794, 2.794){};
            \draw (8.5, 2.5) -| (8.5, 1.25);
        \end{tikzpicture}
    \end{equation*}
\end{minipage}%
\hspace{0.05\textwidth}% espacio entre figuras
\begin{minipage}{0.45\textwidth}
    \centering
    \begin{equation*}
    \begin{tikzpicture}
        \node[shape=rectangle, draw, line width=1pt, minimum width=1.965cm, minimum height=1.965cm] at (5, 5){};
        \draw (4.25, 5.5) -- (4.25, 5.75);
        \draw (4.5, 5.5) -- (4.5, 5.75);
        \draw (4.75, 5.75) -- (4.75, 5.5);
        \draw (4, 5.75) -- (2.5, 5.75);
        \draw (4, 1.75) -- (2.5, 1.75);
        \draw (6, 5.75) -| (8, 5.75);
        \draw (6, 1.75) -| (8, 3.25);
        \node[shape=rectangle, draw, line width=1pt, minimum width=1.965cm, minimum height=1.965cm] at (5, 2.5){};
        \draw (4.25, 3.25) -| (4.25, 3);
        \draw (4.5, 3.25) -| (4.5, 3);
        \draw (4.75, 3.25) -| (4.75, 3);
        \node[oscillator, xscale=0.6, yscale=0.6] at (8.294, 3.544){};
        \draw (3.5, 4.25) to[voltmeter, /tikz/circuitikz/bipoles/length=0.840cm] (3.5, 3.25);
        \draw (6, 4.25) -- (7, 4.25);
        \draw (6, 3.25) -| (7, 4.25);
        \draw (8, 5.75) to[american resistor, /tikz/circuitikz/bipoles/length=0.840cm] (8, 3.838);
        \draw (3.5, 4.25) -- (4, 4.25);
        \draw (3.5, 3.25) -- (4, 3.25);
    \end{tikzpicture}    
\end{equation*}
\end{minipage}
\caption{Test de Brune para el conexionado en Paralelo.}
\label{fig: testBruneParalelo}
\end{figure}

Si se cumple que la tensión del voltímetro es cero o cercana a cero en ambos casos, entonces se cumple la condición de Brune y se puede 
obtener la matriz Y de la manera explicada.


\subsubsection*{Conexión en Cascada}

\begin{figure}
    \begin{equation*}
        \begin{tikzpicture}
        % Paths, nodes and wires:
        \node[shape=rectangle, draw, line width=1pt, minimum width=1.965cm, minimum height=1.965cm] at (5, 5){};
        \draw (4.25, 5.5) -- (4.25, 5.75);
        \draw (4.5, 5.5) -- (4.5, 5.75);
        \draw (4.75, 5.75) -- (4.75, 5.5);
        \draw (4, 5.75) -- (2.5, 5.75);
        \draw (4, 4.239) -- (2.5, 4.239);
        \draw (8.75, 5.75) -- (10.25, 5.75);
        \draw (8.761, 4.261) -- (10.261, 4.261);
        \draw[-latex] (2.5, 6) -- (3.25, 6);
        \draw[-latex] (3.25, 4.5) -- (2.5, 4.5);
        \draw[-latex] (10.25, 6) -- (9.5, 6);
        \draw[-latex] (9.5, 4.533) -- (10.25, 4.533);
        \node[plain crossing] at (1.879, 5.75){};
        \node[plain crossing] at (10.89, 5.75){};
        \node[shape=rectangle, minimum width=0.465cm, minimum height=0.465cm] at (2.5, 6.25){} node[anchor=center, align=center, text width=0.077cm, inner sep=6pt] at (2.5, 6.25){\scriptsize I1};
        \node[shape=rectangle, minimum width=0.465cm, minimum height=0.465cm] at (2.5, 4.739){} node[anchor=center, align=center, text width=0.077cm, inner sep=6pt] at (2.5, 4.739){\scriptsize I1};
        \node[shape=rectangle, minimum width=0.465cm, minimum height=0.465cm] at (10.25, 6.25){} node[anchor=center, align=center, text width=0.077cm, inner sep=6pt] at (10.25, 6.25){\scriptsize I2};
        \node[shape=rectangle, minimum width=0.465cm, minimum height=0.465cm] at (10.25, 4.75){} node[anchor=center, align=center, text width=0.077cm, inner sep=6pt] at (10.25, 4.75){\scriptsize I2};
        \node[shape=rectangle, minimum width=0.465cm, minimum height=0.465cm] at (1.869, 5.106){} node[anchor=center, align=center, text width=0.077cm, inner sep=6pt] at (1.869, 5.106){\scriptsize V1};
        \node[shape=rectangle, minimum width=0.465cm, minimum height=0.465cm] at (10.906, 5.118){} node[anchor=center, align=center, text width=0.077cm, inner sep=6pt] at (10.906, 5.118){\scriptsize V2};
        \node[shape=rectangle, draw, line width=1pt, minimum width=1.965cm, minimum height=1.965cm] at (7.75, 5){};
        \draw (7, 5.75) -| (7, 5.5);
        \draw (7.25, 5.75) -| (7.25, 5.5);
        \draw (7.5, 5.75) -| (7.5, 5.5);
        \draw (2, 4.489) -- (1.75, 4.489);
        \draw (10.75, 4.5) -- (11, 4.5);
        \draw (6, 5.75) -- (6.75, 5.75);
        \draw (6.75, 4.25) -- (6, 4.25);
    \end{tikzpicture}
    \end{equation*}
\caption{Dos cuadripolos conectados en cascada.}
\label{fig: conexinoCascada}
\end{figure}

En este conexionado, se cumple que la matriz transmisión $T$ resultante es
\begin{equation*}
    T = T_A \cdot T_B
\end{equation*}
siendo $ T_A $ e $ T_B $ las matrices de transmisión de los respectivos cuadripolos. Esta relación se cumple siempre para esta forma de conexionado.